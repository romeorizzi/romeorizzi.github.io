\documentclass[10pt]{article}

\usepackage{latexsym}
\usepackage{etaremune}

\textwidth 15.5cm
\textheight 23.2cm
\topmargin -1cm
\evensidemargin 0in
\oddsidemargin 0in

\newcommand{\voice}[1] { \bigskip \medskip \noindent {\Large \bf #1} \medskip\\ }
\newcommand{\subvoice}[1] { {\large \bf #1} \smallskip\\ }
\newcommand{\emp}[1] { {\em #1}\\ }

\begin{document}

\mbox{\ }
\begin{center}
{\Huge \sc {\bf \vspace{-2.2cm} Curriculum Vitae \\ \vspace{4mm}}
               Rizzi Romeo}\\ \vspace{4mm}
               (January 2022)
\end{center}

\mbox{\ }
\vspace{0.5cm}

\voice{{\LARGE Personal Data}}     
%{\sc Marital Status:} married                                       \\
{\sc Nationality:} Italian                                  \\
%{\sc Place of birth:} Mezzolombardo (Trento-Italy)        \\
{\sc Date of birth:} 20 April 1967                              \\
{\sc Private Address:} via Bolleri $N^o$ 16/1 Martignano --- 38121 (TN)   \\
{\sc Phone Number:} (Italy).3291780915 (cel) \ \ \ (Italy).045.802.7088 (office) \\ 
{\sc e-mail:} {\tt Romeo.Rizzi$@$univr.it}               \\
{\sc Home Page:} {\tt http://profs.sci.univr.it/$\sim$rrizzi}  \\


\vspace{1.8mm}

\voice{{\LARGE Research Interests}}

Combinatorial Optimization.
Algorithms.
Computational Biology.
Computational Complexity.
Operations Research.
Approximation Algorithms.
Distributed Algorithms.
Graphs.
Matroids.
Edge Colorings.
Graph Factorization.
Matching Theory and Problems.
Packing and Covering Problems.
Shortest Paths Problems.
Minimum Cuts.\\


\vspace{1.8mm}
\voice{{\LARGE Education}}

\subvoice{Phd in Computational Mathematics and Informatics}
\emp{Department of Mathematics, Padova University.
September 30, 1997.}

\subvoice{Phd Thesis}
{\sc supervisor:} Prof. Michele Conforti (Department of Mathematics,
                Padova University). \\
{\sc external examiner:} Prof. Bert Gerards (CWI Research Institute, Amsterdam). \\
{\sc title:} Packing $T$-cuts and $T$-joins.\\
{\sc interest areas:} Operations Research, Graph theory, Combinatorics.\\


\subvoice{B.Sc.~Degree in Electronics Engineering}
\emp{Politecnico di Milano. 100/100 cum laude. December 20, 1991.}

\subvoice{B.Sc. Thesis}
{\sc supervisor:} Prof. Francesco Maffioli (Electronics Department,   
                Politecnico di Milano), \\
{\sc title:} The $k$-MST Problem.\\
{\sc interest areas:} Operations Research, Combinatorial Optimization.\\


\vspace{1.8mm}

\voice{{\LARGE Present employment}}

\subvoice{Full Professor by the
          University of Verona}
{\bf december 2019 -- today.}
Sector: Operations Research.

{\bf currently in charge by the department:} Membro della Commissione di valutazione assegni di tutorato per corsi di Informatica e Bioinformatica, Membro della Commissione Ammissione Studenti
Internazionali - Matematica, Referente di Dipartimento per le Olimpiadi dell'informatica, Membro del Collegio dei Docenti del Dottorato Interateneo in Matematica, Membro del Collegio Didattico di Informatica, Membro del Collegio Didattico di Matematica, Membro del Consiglio del Dipartimento di Informatica.

{\bf courses for the deparment:} in Verona,
I held the courses ``Ricerca Operativa''
for the bachelor degree in Applied Mathematics L35 (academic years from 2011-12 up to present)
and ``Algoritmi'' for the master degree in Engineering and Informatics LM18+LM32 (academic years from 2011-12 up to present).
I coordinated and held minicourses for the seminar course ``Math Decisions'' (2014-15, 15-16, 16-17, 17-18) for the master degree in Mathematics LM40. 
For the international degree in Mathematics LM40, the course ``Mathematics for Decisions'' (academic years from 2019-20 up to present); prior to this, I coordinated and held minicourses for the seminar course ``Mathematics for Decisions'' (2014-15, 15-16, 16-17, 17-18, 18-19).
Open to all the students of the department, the course ``Programming Challenges'' (academic years from 2013-14 to 2020-21).
I held classes for highschool teachers in TFA (2012-13, 2014-15) and PAS (2013-14, 2014-15).
I have experimented tandem courses (an offert from the University of Verona to high-school studens) in algorithms.

{\bf further didactical activities:}
Since 2001 I am also active as a trainer and tutor for the Olympiads in Informatics.
In this sector, I have a long and intensive record of activities
which range from classes in high-schools (since 2001) to training and coaching
the italian team (since 2004).\\

\vspace{1.8mm}

\voice{{\LARGE Work Experience}}

\subvoice{Associate Professor by the University of Verona}
{\bf december 2011 -- december 2019.}
Sector: Operations Research.

{\bf in charge by the department:} Presidente di Commissione Paritetica, Membro della Commissione di valutazione assegni di tutorato per corsi di Informatica e Bioinformatica, Referente di Dipartimento per le Olimpiadi dell'informatica, Referente del Dipartimento verso il coderDojo in Verona, Membro del Collegio dei Docenti del Dottorato Interateneo in Matematica, Membro del Collegio Didattico di Informatica, Membro del Collegio Didattico di Matematica, Membro del Consiglio di Corso di Tirocinio Formativo Attivo - TFA classe A042- Informatica, Membro del Consiglio del Dipartimento di Informatica.

{\bf courses for the deparment:} see current employment.\\

\subvoice{Associate Professor by the University of Udine}
{\bf october 2005 -- december 2011.}
Sector: Operations Research.
{\bf (classes)} in Udine,
for the faculty of architectur,
I held the classes ``Ricerca Operativa'' (2006-07, 07-08, 08-09, 09-10), ``Matematica II'' (2005-06, 06-07, 07-08, 08-09),
and ``Matematica'' (2010-11, 2011-2012), for the faculty of enginering I held the classes ``Ricerca Operativa'' (2005-06, 06-07, 07-08, 08-09, 09-10, 10-11, 11-12).
% At the PhD school I taught "Computational Complexity".\\


\subvoice{Assistant Research Professor
          by the Faculty of Science
          at the University of Trento - Italy.}
{\bf march 2001 -- october 2005:}
I taught ``Laboratorio di Algoritmi e Strutture Dati'', ``Algoritmi e Strutture Dati'', ``Complessit\`a Computazionale'', at the master degree, and at the PhD school ``Linear Programming'', ``Computational Molecular Biology''.\\



\subvoice{Researcher at I.R.S.T.}
{\bf August 2000 -- February 2001:}
part of the group CBR (Case Based Reasoning, chief: Paolo Avesani)
of the SRA division
(Automated Reasoning Systems, chief: Paolo Traverso) in IRST.
IRST (Istituto Ricerca Scientifica e Tecnologica)
is part of ITC (Istituto Trentino Cultura)
and is located in Trento - Italy.\\


\subvoice{Post-docs and other temporary positions}
{\bf August 99 -- October 99:}
     Assistant Research Professor at BRICS
     of the University of Aarhus (Denmark). 

\noindent
{\bf April 2000 -- June 2000},
{\bf November 99 -- December 99},
{\bf April 99 -- June 99},
{\bf November 98 -- December 98:}
      I held, for 10 months in total,
      a research position on DONET funds
      at the Research Institute CWI in Amsterdam.
      I was part of the PNA group (Probability, Networks, Algorithms),
      lead and supervised by Professors Alexander Schrijver
      and Bert Gerards.

\noindent
{\bf June 98 -- June 99:}
On a post-doc fellowship
from the University of Padua
spent at the Department of Mathematics
of Padua University.
Supervisor: prof.~Michele Conforti.\\

\subvoice{University teaching}
{\bf Second semester 97/98:}
I taught the course ``Programmazione Combinatoria''
at the Department of Mathematics, Trento University.\\

\subvoice{Activity as a programmer}
{\bf June 97 -- April 98:}
I worked as a programmer
for the Department of Mathematics of the Trento University.\\

\subvoice{Teaching assistant for short degree courses}
{\bf Second semester 96/97:}
Teaching assistant for the class in ``Analisi II''
for the short degree course in Informatics and Automatics
in Rovereto (Trento University).\\

\subvoice{Doctoral fellowship}
{\bf November 93 -- November 96:}
I regularly received the fellowship fund
meant for my Dottorato position and activities
in the Department of Mathematics
of the University of Padova 
under the supervision
and scientific responsibility of Prof.~Conforti.\\

\subvoice{High school teaching (after my degrees)}
I thought into regular state high schools
during the following periods:\\

\begin{center}
\begin{tabular}[c]{||c|p{0.80in}|p{1.25in}|p{1.25in}|p{1.25in}||}
 \hline \hline
  year      & period & school & subject & notes\\
 \hline \hline
  89-90     & {\bf whole year} & I.T.I.S. Hensenberger (Monza)
            & (elettrotecnica) (misure elettriche)
            & before my degrees (only evening classes) \\
% from 14-11-89 to 31-8-90
 \hline \hline
  92-93     & from 21/9/92 to 17/10/92
            & I.T.I. Marconi (Rovereto)
            & (informatica industriale) % code: LIV 
              (matematica applicata) % code: LXIV
            & none \\
  92-93     & from 26/10/92 to 14/11/92 & I.T.C. Martini (Mezzolombardo)
            & 038A (fisica) & none \\
 \hline \hline
  93-94     & from 13/10/93 to 18/11/93 & I.T.I.S. Buonarroti (Trento)
            & 035A (elettrotecnica e applicazioni) & none \\
  93-94     & from 12/2/94 to 26/2/94 & I.T.C. Martini (Mezzolombardo)
            & 048A (matematica applicata) & none \\
 \hline \hline
  95-96     & from 22/9/95 to 6/11/95 & I.T.I.S. Buonarroti (Trento)
            & 035A (elettrotecnica e applicazioni) & 1 day off for competitions\\
 \hline \hline
  96-97   & from 17/4/97 to 21/4/97 & I.P.C. Don Milani (Rovereto)
            & 042A (informatica) & none \\
 \hline \hline
  97-98     & {\bf intero anno scolastico} & I.P.C. Battisti (Trento)
            & 047A (matematica) (matematica ed informatica)
            & none \\
% from 15-9-97 to 30-6-98 con proroga to 31-8-98
 \hline \hline
  98-99   & from 17/9/98 all' 1/10/98 & I.T.C.G. Floriani (Riva)
            & 048A (matematica applicata) & none \\
  98-99   & from 11/1/99 to 11/1/99 & I.T.C.G. Fontana (Rovereto)
            & 047A (matematica) & none \\
 \hline \hline
  99-2000   & from 15/1/00 to 31/3/00 & I.T.I.S. Buonarroti (Trento)
            & 047A (matematica) & none \\
 \hline \hline
\end{tabular}
\end{center}



\vspace{1.8mm}

\voice{{\LARGE Military Service}}

{\bf Performed:} \ \ Enlisted: November 16, 1992.
\ Discharged: November 15, 1993.  \\


\vspace{1.8mm}


\voice{{\LARGE Study and Research abroad}}

   \begin{itemize}
%      \item[] {\bf June -- August 1994:}
%            Cambridge. (England).
%            Just to study English.  
      \item[] {\bf November 1995, October 1996:}
            Guest 
            of Prof.~Andr\'{a}s Seb\"o
            at the Laboratoire
            IMAG % Informatique Mathematiques Appliquees Grenoble 
            and Leibniz of the University of Grenoble, France.
      \item[] {\bf November-December 2000, January--February 2003:}
            Guest 
            of Prof.~Pavol Hell
            at the Department of Mathematics
            of the Simon Fraeser University (SFU)
            of Vancouver, Canada;
            of Prof.~Gary MacGillivray
            by the Department of Mathematics
            of the University of Victoria (UV), Canada
            and of Prof.~Rick Brewster
            by the Department of Computer Science
            of the University of Scherbrook (Montreal), Canada.
      \item[] {\bf August 2001:}
            Guest at BRICS (University of Aarhus, Denmark).
      \item[] {\bf November--December 2004:} 
            Guest
            of Prof.~Pablo Moscato
            by the Bioinformatics Center
            of the University of Newcastle, Australia.
            Visited also the Australian National University in Canberra.
      \item[] {\bf September--October 2005:} 
            Guest
            of Prof.~St\'ephane Vialette
            by l'Universit\'e Paris-Sud (Orsay).
      \item[] {\bf December 2005:} 
            Guest
            of Prof.~Guillaume Fertin
            by l'Universit\'e Nantes.
      \item[] {\bf November 2009:} 
            Invited Professor (``Professor Invitee'') 
            by l'Universit\'e Paris-Est - Marne-la-Vall\'ee
            Invited by Prof.~St\'ephane Vialette.
      \item[] {\bf February 2013:} 
            Invited Professor (``Professor Invitee'') 
            by l'Universit\'e Paris-Est - Marne-la-Vall\'ee
            Invited by Prof.~St\'ephane Vialette.
      \item[] {\bf November 2015:} 
            Invited Professor (``Professor Invitee'') 
            by l'Universit\'e Paris-Est - Marne-la-Vall\'ee
            Invited by Prof.~St\'ephane Vialette.
   \end{itemize}

\vspace{1.8mm}

\voice{{\LARGE Seminars}}

I have promoted the results of my research work
giving seminars at the following institutions (not kept updated list):
Laboratoire IMAG of the CNRS in Grenoble (1995).
Laboratoire Leibniz of the University of Grenoble (1996, 2003).
Istituto IASI del C.N.R. in Roma (1997, 1999),
DEIS dell'Universit\`a di Bologna (1997, 1999, 2001),
DSI dell'Universit\`a di Bologna (2008),
Istituto di Ricerca CWI in Amsterdam (1998),
DEIS del Politecnico di Milano (2000, 2003),
Dip.~Matematica ed Informatica Universit\`a di Udine (2006),
Dipartimento di Informatica della Bicocca di Milano (2003, 2006, 2008),
Dip.~Informatica Universit\`a di Verona (2008),
DSMI dell'Universit\`a di Reggio Emilia (2000),
Istituto IRST dell'ITC di Trento (2000, 2001),
Math. Dept. della Simon Freaser University di Vancouver (2000, 2003),
Math. Dept. della University of Victoria (2003),
Dipartimento di Elettronica del Politecnico di Torino (2003),
Engineering Dept. dell'Universit\`a di New Castle (2004).\\


\newpage

\voice{{\LARGE Publications on International Scientific Journals}}

\begin{etaremune}
  \vspace{-3.0mm}
  \item {\sc Giulia Punzi, Alessio Conte, Roberto Grossi, Romeo Rizzi:}
   \newblock Refined Bounds on the Number of Eulerian Tours in Undirected Graphs,
   \newblock {\it Algorithmica}
   \newblock 86(1) (2024) 194--217.

  \item {\sc Matteo Zavatteri, Alice Raffaele, Dario Ostuni, Romeo Rizzi:}
   \newblock An interdisciplinary experimental evaluation on the disjunctive temporal problem,
   \newblock {\it Constraints An Int. J.}
   \newblock 28(1) (2023) 1--12.

  \item {\sc Enrico Angelelli, Renata Mansini, Romeo Rizzi:}
   \newblock Solving the probabilistic profitable tour problem on a line,
   \newblock {\it Optim. Lett.}
   \newblock 17(8) (2023) 1873--1888

  \item {\sc Romeo Rizzi, Stéphane Vialette:}
   \newblock On recognising words that are squares for the shuffle product,
   \newblock {\it Theor. Comput. Sci.}
   \newblock  956 (2023) 111156.

  \item {\sc Federica Arrigoni, Andrea Fusiello, Romeo Rizzi, Elisa Ricci:}
   \newblock Revisiting Viewing Graph Solvability: an Effective Approach Based on Cycle Consistency,
   \newblock {\it IEEE Transactions on Pattern Analysis and Machine Intelligence}
   \newblock  (2022) 1--14.
   \newblock  doi: 10.1109/TPAMI.2022.3212595. Online ahead of print.

  \item {\sc Massimo Cairo, Shahbaz Khan, Romeo Rizzi, Sebastian S. Schmidt, Alexandru I. Tomescu:}
   \newblock Safety in s-t Paths, Trails and Walks,
   \newblock {\it Algorithmica}
   \newblock 84(3) (2022) 719--741.

  \item {\sc Matteo Zavatteri, Romeo Rizzi, Tiziano Villa:}
   \newblock Dynamic controllability of temporal networks with instantaneous reaction,
   \newblock {\it Inf. Sci.}
   \newblock 613 (2022) 932--952.

  \item {\sc Manuel Cáceres, Brendan Mumey, Edin Husic, Romeo Rizzi, Massimo Cairo, Kristoffer Sahlin, Alexandru I. Tomescu:}
   \newblock Safety in Multi-Assembly via Paths Appearing in All Path Covers of a DAG,
   \newblock {\it IEEE ACM Trans. Comput. Biol. Bioinform.}
   \newblock 19(6) (2022) 3673--3684.

  \item {\sc Massimo Cairo, Shahbaz Khan, Romeo Rizzi, Sebastian S. Schmidt, Alexandru I. Tomescu, Elia C. Zirondelli:}
   \newblock A simplified algorithm computing all s-t bridges and articulation points,
   \newblock {\it Discret. Appl. Math.}
   \newblock 305 (2021) 103--108.

  \item {\sc Carlo Combi, Romeo Rizzi, Pietro Sala:}
   \newblock Checking Sets of Pure Evolving Association Rules,
   \newblock {\it Fundam. Informaticae}
   \newblock 178(4) (2021) 283--313.

  \item {\sc Matteo Zavatteri, Carlo Combi, Romeo Rizzi, Luca Viganò:}
   \newblock Consistency checking of STNs with decisions: Managing temporal and access-control constraints in a seamless way,
   \newblock {\it Inf. Comput.}
   \newblock 280 (2021) 104637.

  \item {\sc Laurent Bulteau, Guillaume Fertin, Anthony Labarre, Romeo Rizzi, Irena Rusu:}
   \newblock Decomposing subcubic graphs into claws, paths or triangles,
   \newblock {\it J. Graph Theory}
   \newblock 98(4) (2021) 557--588.

  \item {\sc Sara Giuliani, Zsuzsanna Lipták, Francesco Masillo, Romeo Rizzi:}
   \newblock When a dollar makes a BWT,
   \newblock {\it Theor. Comput. Sci.}
   \newblock 857 (2021) 123--146.

  \item {\sc Vicente Acuña, Roberto Grossi, Giuseppe Francesco Italiano, Leandro Lima, Romeo Rizzi, Gustavo Sacomoto, Marie-France Sagot, Blerina Sinaimeri:}
   \newblock On Bubble Generators in Directed Graphs,
   \newblock {\it Algorithmica}
   \newblock 82(4) (2020) 898--914.

  \item {\sc Carlo Comin, Anthony Labarre, Romeo Rizzi, Stéphane Vialette:}
   \newblock Sorting with forbidden intermediates,
   \newblock {\it Discret. Appl. Math.}
   \newblock  279 (2020) 49--68.

  \item {\sc Massimo Cairo, Carlo Comin, Romeo Rizzi:}
   \newblock Instantaneous reaction-time in dynamic consistency checking of conditional simple temporal networks,
   \newblock {\it J. Log. Algebraic Methods Program.}
   \newblock  113 (2020) 100542.

  \item {\sc Pietro Sala, Carlo Combi, Matteo Mantovani, Romeo Rizzi:}
   \newblock Discovering Evolving Temporal Information: Theory and Application to Clinical Databases,
   \newblock {\it SN Comput. Sci.}
   \newblock 1(3) (2020) 153.

  \item {\sc Edin Husic, Xinyue Li, Ademir Hujdurovic, Miika Mehine, Romeo Rizzi, Veli Mäkinen, Martin Milanic, Alexandru I. Tomescu:}
   \newblock MIPUP: minimum perfect unmixed phylogenies for multi-sampled tumors via branchings and ILP,
   \newblock {\it Bioinform.}
   \newblock 35(5) (2019) 769--777.

  \item {\sc Romeo Rizzi, Alexandru I. Tomescu:}
   \newblock Faster FPTASes for counting and random generation of Knapsack solutions,
   \newblock {\it Inf. Comput.}
   \newblock  267 (2019) 135--144.

  \item {\sc Massimo Cairo, Paul Medvedev, Nidia Obscura Acosta, Romeo Rizzi, Alexandru I. Tomescu:}
   \newblock An Optimal O(nm) Algorithm for Enumerating All Walks Common to All Closed Edge-covering Walks of a Graph,
   \newblock {\it ACM Trans. Algorithms}
   \newblock  15(4) (2019) 48:1--48:17.

  \item {\sc Romeo Rizzi, Massimo Cairo, Veli Mäkinen, Alexandru I. Tomescu, Daniel Valenzuela:}
   \newblock Hardness of Covering Alignment: Phase Transition in Post-Sequence Genomics,
   \newblock {\it IEEE ACM Trans. Comput. Biol. Bioinform.}
   \newblock 16(1) (2019) 23--30.

  \item {\sc Massimo Cairo, Romeo Rizzi:}
   \newblock Dynamic controllability of simple temporal networks with uncertainty: Simple rules and fast real-time execution,
   \newblock {\it Theor. Comput. Sci.}
   \newblock  797 (2019) 2--16.

  \item {\sc Enrico Fraccaroli, Francesco Stefanni, Romeo Rizzi, Davide Quaglia, Franco Fummi:}
   \newblock Network Synthesis for Distributed Embedded Systems,
   \newblock {\it IEEE Trans. on Computers}
   \newblock 67(9) (2018) 1315--1330.
   % DOI:10.1109/TC.2018.2812797.

  \item {\sc Carlo Comin, Romeo Rizzi:}
   \newblock Checking dynamic consistency of conditional hyper temporal networks via mean payoff games: Hardness and (pseudo) singly-exponential time algorithm,
   \newblock {\it Inf. Comput.}
   \newblock 259(3) (2018) 348--374.

  \item {\sc Carlo Comin, Romeo Rizzi:}
    \newblock An Improved Upper Bound on Maximal Clique Listing via Rectangular Fast Matrix Multiplication,
    \newblock {\it Algorithmica}
    \newblock 80(12) (2018) 3525--3562.
   
  \item {\sc Alessio Conte, Roberto Grossi, Andrea Marino, Romeo Rizzi:}
   \newblock Efficient enumeration of graph orientations with sources,
   \newblock {\it Discrete Applied Mathematics}
   \newblock 246 (2018) 22--37.

  \item {\sc Ademir Hujdurovic, Edin Husic, Martin Milanicw, Romeo Rizzi, Alexandru I. Tomescu:}
   \newblock Perfect Phylogenies via Branchings in Acyclic Digraphs and a Generalization of Dilworth's Theorem,
   \newblock {\it ACM Trans. Algorithms}
   \newblock 14(2) (2018) 20:1--20:26.

  \item {\sc Carlo Comin, Romeo Rizzi:}
   \newblock Improved Pseudo-polynomial Bound for the Value Problem and Optimal Strategy Synthesis in Mean Payoff Games,
   \newblock {\it Algorithmica}
   \newblock 77(4) (2017) 995--1021.

  \item {\sc Carlo Comin, Roberto Posenato, Romeo Rizzi:}
   \newblock Hyper temporal networks - A tractable generalization of simple temporal networks and its relation to mean payoff games,
   \newblock {\it Constraints}
   \newblock 22(2) (2017) 152--190.

  \item {\sc Franca Rinaldi, Romeo Rizzi:}
   \newblock Solving the train marshalling problem by inclusion-exclusion,
   \newblock {\it Discrete Applied Mathematics}
   \newblock 217 (2017) 685--690.

  \item {\sc Liliana Alcón, Marisa Gutierrez, István Kovács, Martin Milanic, Romeo Rizzi:}
   \newblock Strong cliques and equistability of EPT graphs,
   \newblock {\it Discrete Applied Mathematics}
   \newblock 203 (2016) 13--25.

  \item {\sc Both Emerite Neou, Romeo Rizzi, Stéphane Vialette:}
   \newblock Permutation Pattern matching in $(213, 231)$-avoiding permutations,
   \newblock {\it Discrete Mathematics \& Theoretical Computer Science}
   \newblock  18(2) (2016)
   
  \item {\sc David Cariolaro, Romeo Rizzi:}
   \newblock On the Complexity of Computing the Excessive $[B]$-Index of a Graph,
   \newblock {\it Journal of Graph Theory}
   \newblock 82(1) (2016) 65--74.

  \item {\sc Stefano Benati, Romeo Rizzi, Craig A. Tovey:}
   \newblock The complexity of power indexes with graph restricted coalitions,
   \newblock {\it Mathematical Social Sciences}
   \newblock 76 (2015) 53--63.

  \item {\sc Romeo Rizzi, Florian Sikora:}
   \newblock Some Results on More Flexible Versions of Graph Motif,
   \newblock {\it Theory Comput. Syst.}
   \newblock 56(4) (2015) 612--629.

  \item {\sc Alexandru I. Tomescu, Travis Gagie, Alexandru Popa, Romeo Rizzi, Anna Kuosmanen, Veli Mäkinen:}
   \newblock Explaining a Weighted DAG with Few Paths for Solving Genome-Guided Multi-Assembly,
   \newblock {\it IEEE/ACM Trans. Comput. Biology Bioinform.}
   \newblock 12(6) (2015) 1345--1354.

  \item {\sc Ferdinando Cicalese, Martin Milanic, Romeo Rizzi:}
   \newblock On the complexity of the vector connectivity problem,
   \newblock {\it Theor. Comput. Sci.}
   \newblock 591 (2015) 60--71.        

  \item {\sc Alberto Caprara, Mauro Dell'Amico, Jos\'e Carlos D\'\i{}az, Manuel Iori, Romeo Rizzi:}
   \newblock Friendly bin packing instances without Integer Round-up Property,
   \newblock {\it Math. Program.}
   \newblock  150(1) (2015) 5--17.

  \item {\sc Laurent Bulteau, Guillaume Fertin, Romeo Rizzi, St\'ephane Vialette:}
   \newblock  Some algorithmic results for [2]-sumset covers,
   \newblock {\it  Inf. Process. Lett.}
   \newblock  115(1) (2015) 1--5.

  \item {\sc Romeo Rizzi, David Cariolaro:}
   \newblock  Polynomial Time Complexity of Edge Colouring Graphs with Bounded Colour Classes,
   \newblock {\it Algorithmica}
   \newblock   69(3) (2014) 494--500.

  \item {\sc Romeo Rizzi, Alexandru I.~Tomescu, Veli M\"akinen:}
   \newblock  On the complexity of Minimum Path Cover with Subpath Constraints for multi-assembly,
   \newblock {\it  BMC Bioinformatics}
   \newblock  15(S--9) (2014) S5.

  \item {\sc Bostjan Bresar, Tanja Gologranc, Martin Milanic, Douglas F. Rall, Romeo Rizzi:}
   \newblock  Dominating sequences in graphs,
   \newblock {\it  Discrete Mathematics}
   \newblock  336 (2014) 22--36.

  \item {\sc Marien Abreu, Domenico Labbate, Romeo Rizzi, John Sheehan:}
   \newblock  Odd 2-factored snarks,
   \newblock {\it  Eur. J. Comb.}
   \newblock  36 (2014) 460--472.

  \item {\sc Guillaume Blin, Paola Bonizzoni, Riccardo Dondi, Romeo Rizzi, Florian Sikora:}
   \newblock  Complexity insights of the Minimum Duplication problem,
   \newblock {\it  Theor. Comput. Sci.}
   \newblock  530 (2014) 66--79.

  \item {\sc Martin Milanic, Romeo Rizzi, Alexandru I.~Tomescu:}
   \newblock  Set graphs. II. Complexity of set graph recognition and similar problems,
   \newblock {\it Theor. Comput. Sci.}
   \newblock   547 (2014) 70--81. 

  \item {\sc Alexandru I.~Tomescu, Anna Kuosmanen, Romeo Rizzi, Veli M\"akinen:}
   \newblock  A novel min-cost flow method for estimating transcript expression with RNA-Seq,
   \newblock {\it BMC Bioinformatics}
   \newblock   14(S-5) (2013) S15. 

  \item {\sc Guillaume Blin, Romeo Rizzi, Florian Sikora, St\'ephane Vialette:}
   \newblock  Minimum Mosaic Inference of a Set of Recombinants,
   \newblock {\it  Int. J. Found. Comput. Sci.}
   \newblock  24(1) (2013) 51--66.

  \item {\sc Romeo Rizzi, Alexandru I.~Tomescu:}
   \newblock  Ranking, unranking and random generation of extensional acyclic digraphs,
   \newblock {\it Inf. Process. Lett.}
   \newblock   113(5--6) (2013) 183--187. 

  \item {\sc Guillaume Blin, Romeo Rizzi, St\'ephane Vialette:}
   \newblock  A Faster Algorithm for Finding Minimum Tucker Submatrices,
   \newblock {\it  Theory Comput. Syst.}
   \newblock  51(3) (2012) 270--281.

  \item {\sc Romeo Rizzi, Luca Nardin:}
   \newblock  Polynomial Time Instances for the IKHO Problem,
   \newblock {\it ISRN Electronics}
   \newblock  2012, 10 pages (2012).

  \item {\sc Giulia Galbiati, Romeo Rizzi, Edoardo Amaldi:}
   \newblock  On the approximability of the minimum strictly fundamental cycle basis problem,
   \newblock {\it Discrete Applied Mathematics}
   \newblock  159(4) (2011) 187--200.

  \item {Marcin Kubica, Romeo Rizzi, St\'ephane Vialette, Tomasz Walen:}
   \newblock Approximation of RNA multiple structural alignment,
   \newblock {\it J. Discrete Algorithms}
   \newblock 9(4) (2011) 365--376.

  \item {\sc Paola Bonizzoni, Gianluca Della Vedova, Riccardo Dondi, Yuri Pirola, Romeo Rizzi:}
   \newblock  Pure Parsimony Xor Haplotyping,
   \newblock {\it IEEE/ACM Transactions on Computational Biology and Bioinformatics}
   \newblock  7(4) (2010) 598--609.

  \item {\sc David Cariolaro, Romeo Rizzi:}
   \newblock  Excessive factorizations of bipartite multigraphs,
   \newblock {\it Discrete Applied Mathematics}
   \newblock  158 (2010) 1760--1766.

  \item {\sc Ga\"elle Brevier, Romeo Rizzi, St\'ephane Vialette:}
   \newblock   Complexity issues in color-preserving graph embeddings,
   \newblock   {\it Theor. Comput. Sci.}
   \newblock    411(4-5) (2010) 716--729.

  \item {\sc Guillaume Fertin, Danny Hermelin, Romeo Rizzi, St\'ephane Vialette:}
   \newblock  Finding common structured patterns in linear graphs,
   \newblock {\it Theor. Comput. Sci.}
   \newblock 411(26--28) (2010) 2475--2486.

  \item {\sc Romeo Rizzi, Pritha Mahata, Luke Mathieson, Pablo Moscato:}
   \newblock  Hierarchical Clustering Using the Arithmetic-Harmonic Cut: Complexity and Experiments,
   \newblock {\it PLoS ONE}
   \newblock 5(12) (2010) .

  \item {\sc Romeo Rizzi:}
   \newblock   Minimum Weakly Fundamental Cycle Bases Are Hard To Find,
   \newblock {\it Algorithmica}
   \newblock 53(3) (2009) 402--424.

  \item {\sc Telikepalli Kavitha, Christian Liebchen, Kurt Mehlhorn, Dimitrios Michail, Romeo Rizzi, Torsten Ueckerdt, Katharina Anna Zweig:}
   \newblock  Cycle bases in graphs characterization, algorithms, complexity, and applications,
   \newblock   {\it Computer Science Review}
   \newblock   3(4) (2009) 199--243.

  \item {\sc Alan A.~Bertossi, Cristina M.~Pinotti, Romeo Rizzi:}
   \newblock  Optimal receiver scheduling algorithms for a multicast problem,
   \newblock {\it Discrete Applied Mathematics}
   \newblock  157(15) (2009) 3187--3197.

  \item {\sc Peter Biro, David Manlove, Romeo Rizzi:}
   \newblock   Maximum weight cycle packing in directed graphs, with application to kidney exchange programs,
   \newblock {\it Discrete Mathematics, Algorithms and Applications}
   \newblock  1(4) (2009) 499--517.

  \item {\sc Guillaume Fertin, Romeo Rizzi, St\'ephane Vialette:}
   \newblock  Finding Occurrences of Protein
              Complexes in Protein-Protein Interaction Graphs,
   \newblock {\it Journal of Discrete Algorithms}
   \newblock 7(1) (2009) 90--101.

  \item {\sc Ekkehard K\"ohler, Christian Liebchen, Gregor W\"unsch, Romeo Rizzi:}
   \newblock  Lower bounds for strictly fundamental cycle bases in grid graphs.    \newblock {\it Networks}
   \newblock 53(2) (2009) 191--205.

  \item {\sc Stefano Benati, Romeo Rizzi:}
   \newblock   The optimal statistical median of a convex set of arrays,
   \newblock {\it Journal of Global Optimization}
   \newblock 44(1) (2009) 79--97.

  \item {\sc Romeo Rizzi:}
   \newblock   Approximating the Maximum $3$-Edge-Colorable Subgraph Problem,
   \newblock {\it Discrete Mathematics}
   \newblock  309(12) (2009) 4164--4168.

  \item {\sc Richard C.~Brewster, Pavol Hell, Romeo Rizzi:}
   \newblock  Oriented star packings,
   \newblock {\it Journal of Combinatorial Theory, Series~B}
   \newblock  98 (2008) 558--576.

  \item {\sc Giuseppe Lancia, R.~Ravi, Romeo Rizzi:}
   \newblock  Haplotyping for Disease Association: A Combinatorial Approach, 
   \newblock {\it IEEE Transactions on Computational Biology and Bioinformatics}
   \newblock  5(2) (2008) 245--251.

  \item {\sc Danny Hermelin, Dror Rawitz, Romeo Rizzi, St\'ephane Vialette:}
   \newblock  The Minimum Substring Cover Problem,
   \newblock {\it Information and Computation}
   \newblock 206(11) (2008) 1303--1312.

  \item {\sc Reuven Cohen, Liran Katzir, Romeo Rizzi:}
   \newblock   On the Trade-off Between Energy and Multicast Efficiency in 802.16e-like Mobile Networks,
   \newblock {\it IEEE Transactions on Mobile Computing}
   \newblock  7(3) (2008) 346--357.
%   \newblock \\ - a previous version,
%                  with only Reuven Cohen and Romeo Rizzi as authors,
%                  was also accepted at Infocom~2006.

  \item {\sc Giuseppe Lancia, Franca Rinaldi, Romeo Rizzi:}
   \newblock  Flipping letters to minimize the support of a string,
   \newblock  {\it International Journal of Foundations of Computer Science}
   \newblock  19(1) (2008) 5--17.

  \item {\sc Guillaume Blin, Cedric Chauve, Guillaume Fertin, Romeo Rizzi, St\'ephane Vialette:}
   \newblock  Comparing Genomes with Duplications: A Computational Complexity Point of View.
   \newblock  {\it IEEE/ACM Trans. Comput. Biology Bioinform.}
   \newblock  4(4) (2007) 523--534.

  \item {\sc Michael Elkin, Christian Liebchen, Romeo Rizzi:}
   \newblock  New length bounds for cycle bases,
   \newblock {\it Information Processing Letters}
   \newblock  104(5) (2007) 186--193.

  \item {\sc Francesco Maffioli, Romeo Rizzi, Stefano Benati:}
   \newblock  Least and most colored bases,
   \newblock {\it Discrete Applied Mathematics}
   \newblock  155(15) (2007) 1958--1970.

  \item {\sc Stephen Finbow, Andrew King, Gary MacGillivray, Romeo Rizzi:}
   \newblock  The firefighter problem for graphs of maximum degree three,
   \newblock {\it Discrete Mathematics}
   \newblock  307(16) (2007) 2094--2105.

  \item {\sc Christian Liebchen, Romeo Rizzi:}
   \newblock  Classes of cycle bases,
   \newblock {\it Discrete Applied Mathematics}
   \newblock  155 (2007) 337--355.
  % gia inserito nel 2006

  \item {\sc Stefano Benati, Romeo Rizzi:}
   \newblock  A mixed integer linear programming formulation
              of the optimal mean/Value-at-Risk portfolio problem,
   \newblock {\it European Journal of Operational Research}
   \newblock  176 (2007) 423--434.
  % gia inserito nel 2006

  \item {\sc Alessandro Mei, Romeo Rizzi:}
   \newblock  Online Permutation Routing in
              Partitioned Optical Passive Star Networks,
   \newblock {\it IEEE Trans. Computers}
   \newblock  55(12) (2006) 1557--1571.

  \item {\sc Alessandro Mei, Romeo Rizzi:}
   \newblock  Hypercube Computations on Partitioned Optical
              Passive Stars Networks,
   \newblock {\it IEEE Trans. Parallel Distrib. Syst.}
   \newblock  17(6) (2006) 497--507.
%   \newblock \\ - a preliminary version appeared in: HiPC 2003, 95--104.

  \item {\sc Romeo Rizzi:}
   \newblock  Acyclically Pushable Bipartite Permutation Digraphs: an algorithm,
   \newblock {\it Discrete Mathematics}
   \newblock 306(12) (2006) 1177--1188.

  \item {\sc Romeo Rizzi, Marco Rospocher:}
   \newblock  Covering partially directed graphs with directed paths,
   \newblock {\it Discrete Mathematics}
   \newblock 306(13) (2006) 1390--1404.

  \item {\sc Giuseppe Lancia, Romeo Rizzi:}
   \newblock  A polynomial case of the parsimony haplotyping problem,
   \newblock {\it Oper. Res. Lett.}
   \newblock  34(3) (2006) 289--295.

  \item {\sc Guillaume Blin, Guillaume Fertin, Romeo Rizzi,
                  St\'ephane Vialette:}
   \newblock What Makes the Arc-Preserving Subsequence Problem Hard?
   \newblock  {\it Transactions on Computational Systems Biology II}
   \newblock  LNCS vol.~3680 (2005) 1--36.
%   \newblock \\ - a draft version of this work appeared on:
%                  International Conference on Computational Science (2) 2005: 860-868.

  \item {\sc Vineet Bafna, Sorin Istrail, Giuseppe Lancia, Romeo Rizzi:}
   \newblock  Polynomial and APX-hard cases of the Individual Haplotyping Problem,
   \newblock {\it Theoretical Computer Science}
   \newblock  335(1) (2005) 109--125.
%   \newblock \\ - a previous related work by the same authors
%               ``SNPs Problems: Complexity and Practical Algorithms''
%              was also accepted at WABI 2002.

  \item {\sc Zhi-Zhong Chen, Tao Jiang, Guohui Lin, Romeo Rizzi,
                  Jianjun Wen, Dong Xu, Ying Xu:}
   \newblock  More Reliable Protein NMR Peak Assignment via Improved $2$-Interval Scheduling,
   \newblock {\it Journal of Computational Biology}
   \newblock 12(2) 2005 129--146.

  \item {\sc Christian Liebchen, Romeo Rizzi:}
   \newblock  A greedy approach to compute a minimum cycle basis
              of a directed graph,
   \newblock {\it Information Processing Letters}
   \newblock  94(3) (2005) 107--112.
   %IPL3233
   %online via ScienceDirect:
   %http://authors.elsevier.com/TrackMyPaper.html?add_art=myarticles&trk_article=IPL3233&trk_mail=on&trk_surname=Liebchen

  \item {\sc Mauro Cettolo, Michele Vescovi, Romeo Rizzi:}
   \newblock  Evaluation of BIC-based algorithms for audio segmentation,
   \newblock {\it Computer Speech \& Language}
   \newblock 19(2) (2005) 147--170.
%   \newblock Volume 19, Issue 2, April (2005) pages 147-170
%   \newblock \\ - a previous work by the same authors
%                ``A DP Algorithm for Speaker Change Detection'',
%               a starting point for this subsequent work,
%               had been accepted at Eurospeech 2003.

  \item {\sc Elia~Ardizzoni, Alan A.~Bertossi, Maria Cristina Pinotti,
                  Shashank Ramaprasad,  Romeo Rizzi,  Madhusudana V.S. Shashanka:}
   \newblock  Optimal Skewed Data Allocation on Multiple Channels with Flat
              Broadcast per Channel,
   \newblock {\it IEEE Transactions on Computers}
   \newblock  54(5) (2005) 558--572.

  \item {\sc A.A. Bertossi, M.C. Pinotti, R. Rizzi, P. Gupta:} % Phalguni Gupta 
   \newblock  Allocating Servers in Infostations for Bounded Simultaneous Requests,
   \newblock {\it Journal of Parallel and Distributed Computing}
   \newblock  64 (2004) 1113--1126.
%   \newblock \\ - also accepted at IEEE Int'l Parallel and Distributed Processing Symposium, 2003.

  \item {\sc Alan A.~Bertossi, Cristina M.~Pinotti, Romeo Rizzi, Anil M.~Shende:}
   \newblock  Channel Assignment for Interference Avoidance in Honeycomb Wireless Networks,
   \newblock {\it Journal of Parallel and Distributed Computing}
   \newblock  64 (2004) 1329--1344.

  \item {\sc Alberto Caprara, Andrea Lodi, Romeo Rizzi:}
   \newblock  On $d$-Threshold Graphs and $d$-Dimensional Bin Packing,
   \newblock {\it Networks}
   \newblock  44(4) (2004) 266--280.
%   \newblock \\ - also accepted at 2003 Optimization Days.

  \item {\sc Alberto Caprara, Alessandro Panconesi, Romeo Rizzi:}
   \newblock  Packing Cuts in Graphs,
   \newblock {\it Networks}
   \newblock  44(1) (2004) 1--11.
%   \newblock \\ - part of this work was published in the proceedings of ESA 2001.

  \item {\sc Giuseppe Lancia, Maria Cristina Pinotti, Romeo Rizzi:}
   \newblock  Haplotyping Populations by Pure Parsimony: Complexity, 
              Exact, and Approximation Algorithms,
   \newblock {\it INFORMS J.~on Comp.}
   \newblock  16(4) (2004) 348--359.

  \item {\sc Michele Conforti, Romeo Rizzi:}  
   \newblock  Combinatorial Optimization
              - Polyhedra and efficiency: A book review,
   \newblock {\it 4OR}
   \newblock 2(2) (2004) 153--159.

  \item {\sc Alberto Caprara, Alessandro Panconesi, Romeo Rizzi:}
   \newblock  Packing Cycles in Undirected Graphs,
   \newblock {\it Journal of Algorithms}
   \newblock  48(1) (2003) 239--256.
%   \newblock \\ - part of this work was published in the proceedings of ESA 2001.
   % report DIT: NO - POLARIS: SI

  \item {\sc Romeo Rizzi:}
   \newblock  On Rajagopalan and Vazirani's $\frac{3}{2}$-Approximation
              Bound for the Iterated $1$-Steiner Heuristic,
   \newblock {\it Information Processing Letters}
   \newblock  86(6) (2003) 335--338.
   %IPL2862
   %online via ScienceDirect:
   %http://www.sciencedirect.com/science?_ob=GatewayURL&_origin=AUTHORALERT&_method=citationSearch&_piikey=S0020019003002102&_version=1&md5=5603f0c51caeab0a09ec923ed91eccbe
   % report DIT: NO - POLARIS: SI

  \item {\sc Alessandro Mei, Romeo Rizzi:}
   \newblock  Routing Permutations in Partitioned Optical Passive Stars Networks,
   \newblock {\it Journal of Parallel and Distributed Computing}
   \newblock  63(9) (2003) 847--852.
   \newblock \\ - also accepted at IPDPS 2002 where it received the {\bf Best Paper Award}.
   % report DIT: NO - POLARIS: SI

  \item {\sc Richard C.~Brewster, Romeo Rizzi:}
   \newblock  On the complexity of digraph packings,
   \newblock {\it Information Processing Letters}
   \newblock  86(2) (2003) 101--106.
   %IPL2829
   %online via ScienceDirect:
   %http://www.sciencedirect.com/science?_ob=GatewayURL&_origin=AUGATEWAY&_method=citationSearch&_piikey=S0020019002004787&_version=1&md5=4550693a5fda4e6a7d92fb51803d3114
   % report DIT: NO - POLARIS: SI

  \item {\sc Romeo Rizzi:}
   \newblock  A Simple Minimum $T$-Cut Algorithm,
   \newblock {\it Discrete Applied Mathematics}
   \newblock  129 (2003) 539--544.
   % report DIT: SI (senza dati rivista) - POLARIS: SI

  \item {\sc Richard C.~Brewster, Pavol Hell, Sarah H.~Pantel, Romeo Rizzi, Anders Yeo:}
   \newblock  Packing paths in digraphs,
   \newblock {\it Journal of Graph Theory}
   \newblock  44(2) (2003) 81--94.
   %Published Online: 2 Sep 2003 DOI: 10.1002/jgt.10126
   % report DIT: NO - POLARIS: SI

  \item {\sc Romeo Rizzi:}
   \newblock  Cycle cover property and $CPP=SCC$ property are not equivalent,
   \newblock {\it Discrete Mathematics}
   \newblock  259 (2002) 337--342.
   % report DIT: SI - POLARIS: SI

  \item {\sc Alberto Caprara, Romeo Rizzi:}
   \newblock  Packing Triangles in Bounded Degree Graphs,
   \newblock {\it Information Processing Letters}
   \newblock  84(4) (2002) 175--180.
   % report DIT: SI - POLARIS: SI

  \item {\sc Romeo Rizzi:}
   \newblock  Minimum $T$-cuts and optimal $T$-pairings,
   \newblock {\it Discrete Mathematics}
   \newblock  257(1) (2002) 177--181.
   % report DIT: SI - POLARIS: SI

  \item {\sc Romeo Rizzi:}
   \newblock  Finding $1$-factors in bipartite regular graphs,
              and edge-coloring bipartite graphs,
   \newblock {\it SIAM Journal on Discrete Mathematics}
   \newblock  15(3) (2002) 283--288. 
   % report DIT: SI - POLARIS: SI

  \item {\sc Alberto Caprara, Romeo Rizzi:}
   \newblock  Improved Approximation for Breakpoint Graph Decomposition
              and Sorting by Reversals,
   \newblock {\it Journal of Combinatorial Optimization}
   \newblock  6 (2002) 157--182.
   % report DIT: SI - POLARIS: SI

  \item {\sc Romeo Rizzi:}
   \newblock  Complexity of Context-free Grammars with Exceptions,
              and the inadequacy of grammars as models for XML and SGML,
   \newblock {\it Markup Languages: Theory and Practice}
   \newblock  3(1) (2001) 107--116.
   % report DIT: SI - POLARIS: SI

  \item {\sc Alessandro Panconesi, Romeo Rizzi:}
   \newblock  Some Simple Distributed Algorithms for Sparse Networks,
   \newblock {\it Distributed Computing}
   \newblock  14 (2001) 97--100.
   % report DIT: SI - POLARIS: SI

  \item {\sc Romeo Rizzi:}
   \newblock  On the Recognition of $P_4$-Indifferent Graphs,
   \newblock {\it Discrete Mathematics}
   \newblock  239 (2001) 161--169.
   % report DIT: SI - POLARIS: SI

  \item {\sc Romeo Rizzi:}
   \newblock  On $4$-connected graphs without even cycle decompositions,
   \newblock {\it Discrete Mathematics}
   \newblock  234 (2001) 181--186.
   % report DIT: SI - POLARIS: SI

  \item {\sc Romeo Rizzi:}
   \newblock  Excluding a simple good pair approach to directed cuts,
   \newblock {\it Graphs and Combinatorics}
   \newblock  17 (2001) 741--744.
   % report DIT: SI - POLARIS: SI

  \item {\sc Michele Conforti, Romeo Rizzi:}  
   \newblock  Shortest Paths in Conservative Graphs,
   \newblock {\it Discrete Mathematics}
   \newblock  226 (2001) 143--153.
%   \newblock \\ - part of this work
%                  was published in the proceedings of AIRO '96.
   % report DIT: SI - POLARIS: SI

  \item {\sc Romeo Rizzi:}
   \newblock  A note on range-restricted circuit covers,
   \newblock {\it Graphs and Combinatorics}
   \newblock  16 (2000) 355--358.
   % report DIT: SI - POLARIS: SI

  \item {\sc Romeo Rizzi:}
   \newblock  On minimizing symmetric set functions,
   \newblock {\it Combinatorica}
   \newblock  20(3) (2000) 445--450.
   % report DIT: SI - POLARIS: SI

  \item {\sc Romeo Rizzi:}
   \newblock  A short proof of K\H{o}nig's matching theorem,
   \newblock {\it Journal of Graph Theory}
   \newblock  33(3) (2000) 138--139.
   % report DIT: SI - POLARIS: SI

  \item {\sc Ajai Kapoor, Romeo Rizzi:}
   \newblock  Edge-coloring bipartite graphs,
   \newblock {\it Journal of Algorithms}
   \newblock  34(2) (2000) 390--396.
   % report DIT: SI - POLARIS: SI

  \item {\sc Romeo Rizzi:}
   \newblock  Indecomposable $r$-graphs and some other counterexamples,
   \newblock {\it Journal of Graph Theory}
   \newblock  32(1) (1999) 1--15.
   % report DIT: SI - POLARIS: SI

  \item {\sc Alberto Caprara, Romeo Rizzi:}  
   \newblock  Improving a Family of Approximation
              Algorithms to Edge Color Multigraphs,
   \newblock {\it Information Processing Letters}
   \newblock  68(1) (1998) 11--15.
   % report DIT: SI - POLARIS: SI

  \item {\sc Romeo Rizzi:}
   \newblock  K\H{o}nig's Edge Coloring Theorem without augmenting paths,
   \newblock {\it Journal of Graph Theory}
   \newblock  29 (1998) 87.

\end{etaremune}


   
\voice{{\LARGE International Conferences with Referee}}


\begin{etaremune}

\item {\sc D.~Ostuni, A.~Raffaele, R.~Rizzi, M.~Zavatteri:}
   \newblock Faster and Better Simple Temporal Problems,
   \newblock  AAAI 2021:
   \newblock  11913--11920 (2021)
   
\item {\sc D.~Ostuni, E.~Morassutto, R.~Rizzi:}
   \newblock Make your programs compete and watch them play in the Code Colosseum,
   \newblock  CoG 2021:
   \newblock  1--5 (2021)
   
\item {\sc M.~Cairo, R.~Rizzi, A.I.~Tomescu, E.C.~Zirondelli:}
   \newblock Genome Assembly, from Practice to Theory: Safe, Complete and Linear-Time,
   \newblock  ICALP 2021:
   \newblock  43:1--43:18 (2021)
   
\item {\sc Manuel Cáceres, M.~Cairo, B.~Mumey, R.~Rizzi, A.I.~Tomescu:}
   \newblock A Linear-Time Parameterized Algorithm for Computing the Width of a DAG,
   \newblock  WG 2021:
   \newblock  257--269 (2021)
   
\item {\sc M.~Zavatteri, R.~Rizzi, T.~Villa:}
   \newblock Dynamic Controllability and $(J,K)$-Resiliency in Generalized Constraint Networks with Uncertainty,
   \newblock  ICAPS 2020:
   \newblock  314--322 (2020)
   
\item {\sc M.~Zavatteri, R.~Rizzi, T.~Villa:}
   \newblock On the Complexity of Resource Controllability in Business Process Management,
   \newblock  Business Process Management Workshops 2020:
   \newblock  168--180 (2020)
   
\item {\sc M.~Zavatteri, R.~Rizzi, T.~Villa:}
   \newblock Temporal Networks with Decisions,
   \newblock  OVERLAY@AI*IA 2019:
   \newblock  77--82 (2019)
   
\item {\sc M.~Zavatteri, R.~Rizzi, T.~Villa:}
   \newblock  and Dynamic Controllability of CNCUs,
   \newblock  OVERLAY@AI*IA 2019:
   \newblock  83--88 (2019)
   
 \item {\sc S.~Giuliani, Z.~Lipták, R.~Rizzi:}
   \newblock  When a Dollar Makes a BWT,
   \newblock  ICTCS 2019:
   \newblock  20--33 (2019)
  
\item {\sc M.~Zavatteri, C.~Combi, R.~Rizzi, L.~Viganò:}
   \newblock Hybrid SAT-Based Consistency Checking Algorithms for Simple Temporal Networks with Decisions,
   \newblock  TIME 2019:
   \newblock  16:1--16:17 (2019)
  
  \item {\sc C.~Comin, R.~Rizzi:}
   \newblock On Restricted Disjunctive Temporal Problems: Faster Algorithms and Tractability Frontier.
   \newblock TIME 2018:
   \newblock 10:1--10:20 (2018)
  
  \item {\sc M.~Cairo, L.~Hunsberger, R.~Rizzi:}
   \newblock Faster Dynamic Controllability Checking for Simple Temporal Networks with Uncertainty,
   \newblock TIME 2018:
   \newblock 8:1--8:16 (2018)
   
  \item {\sc L.~Bulteau, R.~Rizzi, S.~Vialette:}
   \newblock Pattern Matching for $k$-Track Permutations,
   \newblock IWOCA 2018:
   \newblock 102--114 (2018)

  \item {\sc A.~Conte, R.~Grossi, A.~Marino, R.~Rizzi, L.~Versari:}
   \newblock Listing Subgraphs by Cartesian Decomposition,
   \newblock MFCS 2018:
   \newblock 84:1--84:16 (2018)

  \item {\sc A.~Conte, R.~Grossi, A.~Marino, R.~Rizzi, T.~Uno, L.~Versari:}
   \newblock Tight Lower Bounds for the Number of Inclusion-Minimal $st$-Cuts,
   \newblock WG 2018:
   \newblock 100--110 (2018)
   
  \item {\sc M.~Cairo, P.~Medvedev, N.O.~Acosta, R.~Rizzi, A.I.~Tomescu:}
   \newblock Optimal Omnitig Listing for Safe and Complete Contig Assembly,
   \newblock CPM 2017:
   \newblock 29:1--29:12 (2017)

  \item {\sc M.~Cairo, R.~Rizzi:}
   \newblock The Complexity of Simulation and Matrix Multiplication,
   \newblock SODA 2017:
   \newblock 2203--2214 (2017)

  \item {\sc M.~Cairo, R.~Rizzi:}
   \newblock Dynamic Controllability Made Simple,
   \newblock TIME 2017:
   \newblock 8:1--8:16 (2017)

  \item {\sc M.~Cairo, C.~Combi, C.~Comin, L.~Hunsberger, R.~Posenato, R.~Rizzi, M.~Zavatteri:}
   \newblock Incorporating Decision Nodes into Conditional Simple Temporal Networks,
   \newblock TIME 2017:
   \newblock 9:1--9:18 (2017)

  \item {\sc M.~Cairo, L.~Hunsberger, R.~Posenato, R.~Rizzi:}
   \newblock A Streamlined Model of Conditional Simple Temporal Networks - Semantics and Equivalence Results,
   \newblock TIME 2017:
   \newblock 10:1--10:19 (2017)

  \item {\sc V.~Acu\~na, R.~Grossi, G.F.~Italiano, L.~Lima, R.~Rizzi, G.~Sacomoto, M-F. Sagot, B.~Sinaimeri:}
   \newblock On Bubble Generators in Directed Graphs,
   \newblock  WG 2017:
   \newblock  18--31 (2017)

  \item {\sc A.~Hujdurovic, E.~Husic, M.~Milanic, R.~Rizzi, A.I.~Tomescu:}
   \newblock The Minimum Conflict-Free Row Split Problem Revisited,
   \newblock  WG 2017:
   \newblock  303--315 (2017)

  \item {\sc M.~Cairo, G.~Farina, R.~Rizzi:}
   \newblock Decoding Hidden Markov Models Faster Than Viterbi Via Online Matrix-Vector (max, +)-Multiplication,
   \newblock AAAI 2016:
   \newblock 1484--1490 (2016)

  \item {\sc C.~Comin, A.~Labarre, R.~Rizzi, S.~Vialette:}
   \newblock Sorting with Forbidden Intermediates,
   \newblock AlCoB 2016:
   \newblock 133--144 (2016)

  \item {\sc A.~Kuosmanen, A.~Sobih, R.~Rizzi, V.~Mäkinen, A.I.~Tomescu:}
   \newblock On using Longer RNA-seq Reads to Improve Transcript Prediction Accuracy,
   \newblock BIOINFORMATICS 2016:
   \newblock 272--277 (2016)

  \item {\sc L.~Bulteau, G.~Fertin, A.~Labarre, R.~Rizzi, I.~Rusu:}
   \newblock Decomposing Cubic Graphs into Connected Subgraphs of Size Three,
   \newblock COCOON 2016:
   \newblock 393--404 (2016)

  \item {\sc A.~Conte, R.~Grossi, A.~Marino, R.~Rizzi, L.~Versari:}
   \newblock Directing Road Networks by Listing Strong Orientations,
   \newblock IWOCA 2016:
   \newblock 83--95 (2016)

  \item {\sc A.~Conte, R.~Grossi, A.~Marino, R.~Rizzi:}
   \newblock Listing Acyclic Orientations of Graphs with Single and Multiple Sources,
   \newblock LATIN 2016:
   \newblock 319--333 (2016)

  \item {\sc A.~Farinelli, G.~Franco, R.~Rizzi:}
   \newblock Minimal Multiset Grammars for Recurrent Dynamics,
   \newblock Int. Conf. on Membrane Computing 2016:
   \newblock 177--189 (2016)

  \item {\sc M.~Cairo, R.~Grossi, R.~Rizzi:}
   \newblock New Bounds for Approximating Extremal Distances in Undirected Graphs,
   \newblock SODA 2016:
   \newblock 363--376 (2016)

  \item {\sc B.E.~Neou, R.~Rizzi, S.~Vialette:}
   \newblock Pattern Matching for Separable Permutations,
   \newblock SPIRE 2016:
   \newblock 260--272 (2016)

  \item {\sc M.~Cairo, C.~Comin, R.~Rizzi:}
   \newblock Instantaneous Reaction-Time in Dynamic-Consistency Checking of Conditional Simple Temporal Networks,
   \newblock TIME 2016:
   \newblock 80--89 (2016)

  \item {\sc M.~Cairo, R.~Rizzi:}
   \newblock Dynamic Controllability of Conditional Simple Temporal Networks Is PSPACE-complete,
   \newblock TIME 2016:
   \newblock 90--99 (2016)

  \item {\sc A.~Conte, R.~Grossi, A.~Marino, R.~Rizzi:}
   \newblock Enumerating Cyclic Orientations of a Graph,
   \newblock IWOCA 2015:
   \newblock 88--99 (2015)

  \item {\sc C.~Comin, R.~Rizzi:}
   \newblock Dynamic Consistency of Conditional Simple Temporal Networks via Mean Payoff Games: A Singly-Exponential Time DC-checking,
   \newblock TIME 2015:
   \newblock 19--28 (2015)

  \item {\sc C.~Combi, R.~Rizzi, P.~Sala:}
   \newblock The Price of Evolution in Temporal Databases,
   \newblock TIME 2015:
   \newblock 47--58 (2015)

  \item {\sc R.~Rizzi, G.~Sacomoto, M-F.~Sagot:}
   \newblock Efficiently Listing Bounded Length st-Paths,
   \newblock IWOCA 2014:
   \newblock 318--329 (2014)

  \item {G.~Bacci, M.~Miculan, R.~Rizzi:}
   \newblock Finding a Forest in a Tree - The Matching Problem for Wide Reactive Systems,
   \newblock TGC 2014:
   \newblock 17--33 (2014)

  \item {R.A.~Ferreira, R.~Grossi, R.~Rizzi, G.~Sacomoto, M-F.~Sagot:}
   \newblock Amortized $\tilde{O}(|V|)$-Delay Algorithm for Listing Chordless Cycles in Undirected Graphs,
   \newblock ESA 2014:
   \newblock 418--429 (2014)

  \item {R.~Rizzi, A.I.~Tomescu:}
   \newblock Faster FPTASes for Counting and Random Generation of Knapsack Solutions,
   \newblock ESA 2014:
   \newblock 762--773 (2014)

  \item {G.~Blin, P.~Morel, R.~Rizzi, S.~Vialette:}
   \newblock Towards Unlocking the Full Potential of Multileaf Collimators,
   \newblock SOFSEM 2014:
   \newblock 138--149 (2014)

  \item {C.~Comin, R.~Rizzi, R.~Posenato:}
   \newblock A Tractable Generalization of Simple Temporal Networks and Its Relation to Mean Payoff Games,
   \newblock TIME 2014:
   \newblock 7--16 (2014)

  \item {R.~Rizzi, S.~Vialette:}
   \newblock On Recognizing Words That Are Squares for the Shuffle Product,
   \newblock CSR 2013: 8th International Computer Science Symposium in Russia.
   \newblock LNCS~7913: 235--245 (2013)

  \item {E.~Birmel\'e, R.A.~Ferreira, R.~Grossi, A.~Marino, N.~Pisanti, R.~Rizzi, G.~Sacomoto:}
   \newblock Optimal Listing of Cycles and st-Paths in Undirected Graphs,
   \newblock SODA 2013:
   \newblock SIAM: 1884--1896 (2013)

  \item {F.~Cicalese, T.~Gagie, E.~Giaquinta, E.S.~Laber, Z.~Lipt\'ak, R.~Rizzi, A.I.~Tomescu:}
   \newblock Indexes for Jumbled Pattern Matching in Strings, Trees and Graphs,
   \newblock SPIRE 2013:
   \newblock 56--63 (2013)

  \item {R.~Rizzi, R.~Posenato:}
   \newblock Optimal Design of Consistent Simple Temporal Networks,
   \newblock TIME 2013:
   \newblock 19--25 (2013)

  \item {A.I.~Tomescu, A.~Kuosmanen, R.~Rizzi, V.~M\"akinen:}
   \newblock A Novel Combinatorial Method for Estimating Transcript Expression with RNA-Seq: Bounding the Number of Paths,
   \newblock WABI 2013:
   \newblock 85--98 (2013)

\vspace{-1.8mm}
  \item {R.~Rizzi, F.~Sikora:}
   \newblock Some Results on more Flexible Versions of Graph Motif,
   \newblock CSR 2012: 7th International Computer Science Symposium in Russia.
   \newblock LNCS~7353: 278--289 (2012)

\vspace{-1.8mm}
  \item {D.~Hermelin, R.~Rizzi, S.~Vialette:}
   \newblock Algorithmic Aspects of the Intersection and Overlap Numbers of a Graph,
   \newblock ISAAC 2012: Algorithms and Computation - 23rd International Symposium
   \newblock LNCS~7676: 465--474 (2012)

\vspace{-1.8mm}
  \item {G.~Blin, P.~Bonizzoni, R.~Dondi, R.~Rizzi, F.~Sikora:}
   \newblock Complexity Insights of the Minimum Duplication Problem,
   \newblock SOFSEM 2012: Theory and Practice of Computer Science.
   \newblock LNCS~7147: 153--164 (2012)

\vspace{-1.8mm}
  \item {X.~Yang, F.~Sikora, G.~Blin, S.~Hamel, R.~Rizzi, S.~Aluru:}
   \newblock An Algorithmic View on Multi-Related-Segments: A Unifying Model for Approximate Common Interval,
   \newblock TAMC 2012: Theory and Applications of Models of Computation - 9th Annual Conference.
   \newblock LNCS~7287: 319--329 (2012)

\vspace{-1.8mm}
  \item {G.~Blin, R.~Rizzi, S.~Vialette:}
   \newblock A Polynomial-Time Algorithm for Finding a Minimal Conflicting Set Containing a Given Row,
   \newblock CSR 2012: 6th International Computer Science Symposium in Russia.
   \newblock LNCS~6651: 373--384 (2011)

\vspace{-1.8mm}
  \item {E.~Amaldi, C.~Iuliano, R.~Rizzi:}
   \newblock On cycle bases with limited edge overlap,
   \newblock CTW 2011: Proceedings of the 10th Cologne-Twente Workshop on graphs and combinatorial optimization
   \newblock http://ctw2011.dia.uniroma3.it/ctw\_proceedings.pdf\#page=64: 52--55 (2011)

\vspace{-1.8mm}
  \item {R.A.~Ferreira, R.~Grossi, R.~Rizzi:}
   \newblock Output-Sensitive Listing of Bounded-Size Trees in Undirected Graphs,
   \newblock 19th Annual European Symposium on Algorithms (ESA 2011).
   \newblock LNCS~6942: 275--286 (2011)

\vspace{-1.8mm}
  \item {E.~Amaldi, C.~Iuliano, R.~Rizzi:}
   \newblock Efficient Deterministic Algorithms for Finding a Minimum Cycle Basis in Undirected Graphs,
   \newblock Integer Programming and Combinatorial Optimization, 14th International Conference (IPCO 2010)
   \newblock LNCS~6080: 397--410 (2010)
%
%\vspace{-1.8mm}
%  \item {P.~Bonizzoni, G.~Della Vedova, R.~Dondi, Y.~Pirola, R.~Rizzi:}
%   \newblock Pure Parsimony Xor Haplotyping,
%   \newblock  CoRR abs\/1001.1210 (2010)

\vspace{-1.8mm}
  \item {G.~Blin, R.~Rizzi, S.~Vialette:}
   \newblock A Faster Algorithm for Finding Minimum Tucker Submatrices,
   \newblock 6th Conference on Computability in Europe (CiE 2010).
   \newblock LNCS~6158: 69--77 (2010)

\vspace{-1.8mm}
  \item {E.~Amaldi, C.~Iuliano, T.~Jurkiewicz, K.~Mehlhorn, R.~Rizzi:}
   \newblock Breaking the $O(m^2n)$ Barrier for Minimum Cycle Bases,
   \newblock 17th Annual European Symposium on Algorithms (ESA 2009).
   \newblock LNCS~5757: 301--312 (2009)

\vspace{-1.8mm}
  \item {P.~Bonizzoni, G.~Della Vedova, R.~Dondi, Y.~Pirola, R.~Rizzi:}
   \newblock Pure Parsimony Xor Haplotyping,
   \newblock Bioinformatics Research and Applications, 5th International Symposium (ISBRA 2009).
   \newblock LNCS~5542: 186--197 (2009)

\vspace{-1.8mm}
  \item {G.~Fertin, D.~Hermelin, R.~Rizzi, S.~Vialette:}
   \newblock Common Structured Patterns in Linear Graphs: Approximations and Combinatorics,
   \newblock 18th Symposium on Combinatorial Pattern Matching (CPM'07).
   \newblock LNCS~4580: 241--252 (2007)

\vspace{-1.8mm}
  \item {G.~Brevier, R.~Rizzi, S.~Vialette:}
   \newblock Pattern Matching in Protein-Protein Interaction Graphs,
   \newblock 16th International Symposium on Fundamentals
              of Computation Theory (FCT 2007).
   \newblock LNCS~4639: 137--148 (2007)

\vspace{-1.8mm}
  \item {D.~Hermelin, D.~Rawitz, R.~Rizzi, S.~Vialette:}
   \newblock The Minimum Substring Cover Problem,
   \newblock In, Christos Kaklamanis, Martin Skutella, editors,
   \newblock 5th Workshop on Approximation and Online Algorithms (WAOA'07).
   \newblock LNCS~4927: 170--183 (2007)

\vspace{-1.8mm}
  \item {C.~Liebchen, G.~W\"unsch, E.~K\"ohler, A.~Reich, R.~Rizzi:}
   \newblock Benchmarks for Strictly Fundamental Cycle Bases,
   \newblock WEA 2007:
   \newblock LNCS~4525: 365--378 (2007)

\vspace{-1.8mm}
  \item {M.~Kubica, R.~Rizzi, S.~Vialette, T.~Wale\'n:}
   \newblock Approximation of RNA Multiple Structural Alignment,
   \newblock 17th Symposium on Combinatorial Pattern Matching (CPM'06).
   \newblock LNCS~4009: 211--222 (2006)

\vspace{-1.8mm}
  \item {G.~Lancia, F.~Rinaldi, R.~Rizzi:}
   \newblock Flipping letters to minimize the support of a string,
   \newblock in The Prague Stringology Conference, PSC06,
   \newblock Stringology 9--17 (2006)

\vspace{-1.8mm}
  \item {\sc R.~Cohen, R.~Rizzi:}
   \newblock   On the Trade-Off Between Energy and Multicast Efficiency in 802.16e-Like Mobile Networks,
   \newblock INFOCOM 2006.

\vspace{-1.8mm}
  \item {C.~Chauve, G.~Fertin, R.~Rizzi, S.~Vialette:}
   \newblock Genomes containing duplicates are hard to compare,
   \newblock Int. Workshop on Bioinformatics Research and Applications (IWBRA).
   \newblock LNCS~3992: 783--790 (2006)

\vspace{-1.8mm}
  \item {M.~Dalpasso, G.~Lancia and R.~Rizzi:}
   \newblock The String Barcoding Problem is NP-Hard,
   \newblock in RECOMB Satellite on Comparative Genomics, (A.~Mc Lyshag and D.~Huson eds),
   \newblock Lecture Notes in Bioinformatics, Springer, 85--93, (2005) 

\vspace{-1.8mm}
  \item {G.~Fertin, R.~Rizzi, S.~Vialette:}
   \newblock Finding Exact and Maximum Occurrences
of Protein Complexes in Protein-Protein Interaction Graphs,
   \newblock International Symposium on Mathematical Foundations of Computer Science (MFCS'05).
   \newblock LNCS~3618: 328--339 (2005)

\vspace{-1.8mm}
  \item {A.~Mei, R.~Rizzi:}
   \newblock Online Permutation Routing in Partitioned Optical Passive Star Networks,
   \newblock CoRR abs/cs/0502093. (2005)

\vspace{-1.8mm}
  \item {G.~Blin, G.~Fertin, R.~Rizzi, S.~Vialette:}
   \newblock What Makes the Arc-Preserving Subsequence Problem Hard?
   \newblock 5th Int. Workshop on Bioinformatics Research and Applications (IWBRA'05).
   \newblock  LNCS~3515: 860--868 (2005)

\vspace{-1.8mm}
  \item {G.~Blin, R.~Rizzi:}
   \newblock Conserved Interval Distance Computation Between Non-trivial Genomes,
   \newblock COCOON 2005:
   \newblock  LNCS~3595: 22--31 (2005)

\vspace{-1.8mm}
  \item {G.~Lancia, R.~Rizzi:}
   \newblock Combinatorial Problems Arising in the Analysis of Human Polymorphisms,
   \newblock AIRO 2005, Camerino (2005)

\vspace{-1.8mm}
  \item {G.~Lancia, F.~Rinaldi, R.~Rizzi:}
   \newblock Reducing the k-mer diversity of a string,
   \newblock AIRO 2004, Lecce (2004)

\vspace{-1.8mm}
  \item {G.~Blin, G.~Fertin, R.~Rizzi, S.~Vialette:}
   \newblock Pattern Matching in Arc-Annotated Sequences: New Results for the APS Problem,
   \newblock 5th Journ\'ees Ouvertes de Biologie, Informatique et Math\'ematiques (JOBIM'04).
   \newblock Montr\'eal, Quebec. 2004.
   \newblock IEEE Computer Society.

\vspace{-1.8mm}
  \item {E.~Ardizzoni, A.A.~Bertossi, M.C.~Pinotti, R.~Rizzi:}
   \newblock Comparing Algorithms for Data Broadcasting over Multiple Channels,
   \newblock Algorithms for Wirelss and Ad-hoc networks (ASWAN),
   \newblock Boston, USA, August 2004.

\vspace{-1.8mm}
  \item {A.A.~Bertossi, M.C.~Pinotti, S.~Ramaprasad, R.~Rizzi, M.V.S.~Shashanka:}
   \newblock Optimal multi-channel data allocation with flat broadcast per channel,
   \newblock IEEE Int’l Parallel and Distributed Processing Symposium (IPDPS),
   \newblock Santa Fe, April 2004.

\vspace{-1.8mm}
  \item {Z-Z.~Chen, T.~Jiang, G.-H. Lin, R.~Rizzi, J.~Wen, D.~Xu, Y.~Xu:}
   \newblock More Reliable Protein NMR Peak Assignment via Improved 2-Interval Scheduling,
   \newblock ESA 2003:
   \newblock LNCS~2832: 580-592 (2003)

\vspace{-1.8mm}
  \item {M.~Cettolo, M.~Vescovi, R.~Rizzi:}
   \newblock  A DP Algorithm  for Speaker Change Detection,
   \newblock Eurospeech 2003.

\vspace{-1.8mm}
  \item {A.~Mei, R.~Rizzi:}
   \newblock Mapping Hypercube Computations onto Partitioned Optical Passive Star Networks,
   \newblock HiPC:
   \newblock  LNCS~2913: 95--104 (2003)

\vspace{-1.8mm}
  \item {S.~Finbow, A.~King, G.~MacGillivray, R.~Rizzi:}
   \newblock The Firefighter Problem for Graphs of Maximum Degree Three,
   \newblock EuroComb'03.

\vspace{-1.8mm}
  \item {A.A.~Bertossi, M.C.~Pinotti, R.~Rizzi, A.M.~Shende:}
   \newblock Channel Assignment in Honeycomb Networks,
   \newblock 3rd ICTCS,
   \newblock Bertinoro, Italy, October 2003.

\vspace{-1.8mm}
  \item {A.A.~Bertossi, M.C.~Pinotti R.~Rizzi:}
   \newblock Channel Assignment with Separation on Trees and Interval Graphs,
   \newblock 3rd Int’l Workshop on Wireless, Mobile and Ad Hoc Networks,
   \newblock (satellite workshop of IEEE IPDPS 2003),
   \newblock April 2003.

\vspace{-1.8mm}
  \item {A.A.~Bertossi, M.C.~Pinotti, R.~Rizzi, P.~Gupta:}
   \newblock Allocating Servers in Infostations for Bounded Simultaneous Requests,
   \newblock IEEE Int’l Parallel and Distributed Processing Symposium (IPDPS),
   \newblock Nice, April 2003.

\vspace{-1.8mm}
  \item {A.A.~Bertossi, M.C.~Pinotti, R.~Rizzi:}
   \newblock Channel assignment on strongly-simplicial graphs,
   \newblock IEEE WMAN,
   \newblock Nice, April 2003.

\vspace{-1.8mm}
  \item {R.~Rizzi, V.~Bafna, S.~Istrail, and G.~Lancia:}
   \newblock Practical Algorithms and Fixed-parameter Tractability
   \newblock of the Single Individual SNP Haplotyping Problem,
   \newblock 2nd Workshop on Algorithms in Bioinformatics (WABI),
   \newblock LNCS 2452: 29--43, 2002.

\vspace{-1.8mm}
  \item {A.~Mei, R.~Rizzi:}
   \newblock Routing Permutations in Partitioned Optical Passive Star Networks,
   \newblock IPDPS 2002.

\vspace{-1.8mm}
  \item {A.~Caprara, A.~Panconesi, R.~Rizzi:}
   \newblock  Packing Cycles and Cuts in Undirected Graphs,
   \newblock ESA 2001:
   \newblock LNCS 2161: 512--523 (2001)

\vspace{-1.8mm}
  \item {R.~Rizzi:}
   \newblock On minimizing symmetric set functions,
   \newblock Fourth Slovene International Conference
             in Graph Theory, 1999.

\vspace{-1.8mm}
  \item {M.~Conforti, R.~Rizzi:}
   \newblock Shortest Paths in Conservative Graphs,
   \newblock AIRO '96, Perugia, 1996.

\end{etaremune}




\vspace{1.8mm}

\voice{{\LARGE Chapters in Books}}

\input{ListaCapitoliLibro}


\vspace{1.8mm}

\voice{{\LARGE Books (in Italian, for didactic purposes)}}

\input{ListaLibri}



\vspace{1.8mm}


%\voice{{\LARGE Software Projects}}
%
%\begin{itemize}
%\vspace{-3.0mm}
%  \item[1.] {\sc Romeo Rizzi},
%   \newblock  {\em A simple snark recognizer}.
%   \newblock this is a ``branch and cut'' project under ABACUS, to recognize
%             $3$-graphs whose edges can not be colored with $3$ colors.
%
%  \item[2.] A main theme of LEA group in the Department of Mathematics
%            of Trento University
%            is the realization of ``intertools''
%            available under WEB for the heuristic solution
%            of NP-complete problems.
%            Within such project,
%            I developed the intertool for the
%            ``graph partitioning'' problem.
%
%  \item[3.] {\em Giornalino Virtuale}.
%            This is a project of the Free University of Bozen
%            which has lead the pupils of the several primary schools
%            of Bozen to design, project and create their journal
%            in the web.
%            I have mainly participated to this project
%            in the quality of technician,
%            with the role to assist on informatic issues.
%\end{itemize}

\voice{{\LARGE Honors}}

   - Since 2004, acting as Area Editor for the scientific journal 4OR.

   - Editor of the proceedings of the Oberwolfach meeting
   in Graph Theory, January 2003,
   organized by Reinhard Diestel, Alexander Schrijver
   and Paul D. Seymour.

   - Wrote, together with Prof.~Michele Conforti,
   the review on {\em 4OR} of the masterpiece
   {\em ``Combinatorial Optimization
              - Polyhedra and efficiency''}
   of Alexander Schrijver.

   - Best Paper Award at IPDPS 2002
     for a joint work with Alessandro Mei.

   - Held the class
    ``Algorithmic and Complexity issues in Structure Prediciton and/or Determination''
     at the Third International School on Biology,
     Computation and Information (BCI 2006).
     Dobbiaco (BZ), Italy, September 11-15, 2006.
    %home page della scuola: http://bioinf.dimi.uniud.it/bci2006/

   - Organizer, together with Giuseppe Lancia,
     of an invited session in Computational Biology
     ad AIRO 2005. Camerino.

   - Invited speaker at BioInfoSummer 2004,
     held by the Australian National University in Canberra.

   - Invited speaker at
     ``Workshop on Cycle and Cut Bases'' (2008)
     held at T\"ubingen and inserted in the framework
     SPP 1126 (Algorithmik großer und komplexer Netzwerke).

   - Reviewer of Mathematical Reviews for the American Mathematical Society
     since 2004.

   - Biography included in the 2007 and 2009
     editions of \emph{Who's Who in the World}.

   - Biography included in the 2009/2010 edition
     of \emph{Outstanding Intellectuals of the 21st Century} (IBC, Cambridge).

   - Erd\"os number: 2.

   - As educator, received the following recognizements
     from the \emph{International Biographical Center, Cambridge}:

   \ \ \ - \emph{The Decree of Excellence in Education}.

   \ \ \ - \emph{International Educator of the Year} for 2007 and 2009.

   \ \ \ - \emph{Top 100 Educators} 2008 and 2009.


\vspace{1.2cm}
Verona \hspace{7.8cm} Romeo Rizzi

\end{document}












