\documentclass[10pt]{article}

\usepackage{latexsym}
\usepackage{etaremune}
\usepackage[italian]{babel}
\usepackage[T1]{fontenc}
\usepackage[utf8]{inputenc}
%\usepackage[latin1]{inputenc}


\hyphenation{ra-gio-na-men-to}
\hyphenation{a-stra-zio-ne}
\hyphenation{pro-se-gui-men-to}


\textwidth 15.5cm
\textheight 23.8cm
\topmargin -1.4cm
\evensidemargin 0in
\oddsidemargin 0in

\newcommand{\voice}[1] { \bigskip \medskip \noindent {\Large \bf #1} \medskip\\ }
\newcommand{\subvoice}[1] { {\large \bf #1} \smallskip\\ }
\newcommand{\emp}[1] { {\em #1}\\ }

\begin{document}


\mbox{\ }
\vspace{-2cm} 
\begin{center}
{\Huge \sc {\bf Curriculum Vitae \\ \vspace{4mm}}
               Rizzi Romeo}\\ \vspace{4mm}
               (dicembre 2022)
\end{center}

\mbox{\ }
\vspace{-0.1cm}

\voice{{\LARGE Dati personali}}
Nato a Mezzolombardo (Trento),
il 20 aprile 1967.\\     
%{\sc Cittadinanza:} italiana                               \\
%{\sc Data e Luogo di nascita:} 20/4/67, Mezzolombardo (Trento) \\
{\sc Residenza:} via Bolleri $N^o$ 16/1 Trento (Martignano) --- 38121 (TN)   \\
{\sc Telefono:} 3518684000 (cel)   \\ 
{\sc e-mail:} {\tt Romeo.Rizzi$@$univr.it}               \\
{\sc Home Page:} {\tt http://profs.sci.univr.it/$\sim$rrizzi}  \\


\vspace{0.4cm}

\voice{{\LARGE Interessi ed Aree di Ricerca}}

{\bf Ricerca Operativa.}
    Problemi di basi di cicli, di cammini minimi, di taglio minimo, e di routing.
    Problemi di selezione di portafoglio.
    Problemi di power management.

{\bf Biologia Computazionale.}
    Haplotyping di individui e di popolazioni.
    NMR peack analysis.
    Confronto ed analisi di stringhe con struttura.
    Individuazione di Motifs.
    Algoritmica di permutazioni motivata da genomica.
    Rilevamento di pathways.
    Protein design.
    Raffinamento di modelli per la lettura dei contigs.
    Problematiche di ottimizzazione in radioterapia. 

{\bf Ottimizzazione Combinatoria.}
    Grafi,
    Matroidi,
    Colorazione di archi,
    Fattorizzazione di grafi,
    Teoria e problemi di matching,
    Basi di cicli,
    Problemi di packing e covering,
    Problemi di channel assignment.

{\bf Algoritmi.}
    Algoritmi Polinomiali e Pseudopolinomiali,
    Algoritmi Approssimati,
    PTAS e FPTAS,
    Algoritmi Distribuiti,
    Algoritmi Paralleli,
    Algoritmi Randomizzati,
    Algoritmi Euristici.
    Algoritmi efficienti per il Listing e l'Enumerazione.
    Controllo dinamico discreto.
    Reti temporali e workflows.
    Giochi combinatorici legati a problematiche di model checking.

{\bf Complessit\`a Computazionale.}
    Risultati di NP-completezza,
    Buone caratterizzazioni,
    Risultati di APX-hardness ed inapprossimabilit\`a,
    Risultati di PSPACE-completezza.
    FPT e W-hardness.
    Lower bounds condizionali.

{\bf Calcolo ed Architetture Parallele.}
    Analisi delle POPS-networks.

{\bf Reti.}
    Problemi di frequency assignment e di channel assignment.
    Problemi di scheduling.\\



\voice{{\LARGE Titoli di Studio}}

\vspace{-0.8mm}
\medskip \noindent
(giugno 1986) \
{\large \bf Diploma di Maturit\`a Scientifica}\\
{\em Liceo Scientifico ``Leonardo da Vinci'', Trento.}\\

\medskip \noindent
(dicembre 1991) \
{\large {\bf Laurea in Ingegneria Elettronica}  ad indirizzo matematico fisico}\\
{\em Politecnico di Milano. Media esami: 29/30. Voto finale: 100/100 e Lode.}

Tesi di Laurea:
{\em ``Il problema dell'albero minimo di cardinalit\`a $k$.''}
Relatore:
{\em prof.~F.~Maffioli (Politecnico di Milano, Dipartimento di Elettronica).}
Aree di interesse:
{\em Ricerca Operativa, Ottimizzazione Combinatoria.}\\

\medskip \noindent
(settembre 1997) \
{\large \bf Dottore di Ricerca}\\
{\em Dottorato in Matematica Computazionale ed Informatica Matematica, IX ciclo.
Dipartimento di Ma\-te\-ma\-ti\-ca Applicata dell'Universit\`a di Padova}.

Tesi di Dottorato:
{\em ``Impaccando $T$-tagli e $T$-giunti.''}
Relatore:
{\em prof.~M.~Conforti (Universit\`a di Padova,
                Dipartimento di Ma\-te\-ma\-ti\-ca).}
Controrelatore:
{\em prof.~A.M.H.~Gerards (Istituto di Ricerca CWI, Amsterdam).}
Aree di interesse:
{\em Ricerca Operativa, Teoria dei grafi, Combinatorica.}\\

\newpage

\voice{{\LARGE Qualifiche e impegni lavorativi attuali}}

\subvoice{Professore ordinario presso la
          Facolt\`a di Scienze MM.FF.NN di Verona}
{\bf dicembre 2019 -- oggi.}
Settore MAT/09 (Ricerca Operativa) - abilitazione conseguita il 20/12/2013.


\subvoice{Altre Attivit\`a}
\indent
{\bf (impegno nelle olimpiadi di informatica)}
sia a livello nazionale, dove collaboro all'allenare la squadra italiana
ed a portare avanti le progettualit\`a collegate alle iOi ed alle Oii,
sia a livello di realt\`a locali in Trento, Bolzano, Verona, Udine.
Sul tavolo delle olimpiadi posso vantare un lungo e sostanziale impegno in varie attivit\`a e progetti. 


\vspace{1.8mm}

\voice{{\LARGE Esperienze di lavoro}}

\subvoice{Professore associato presso la
          Facolt\`a di Scienze MM.FF.NN di Verona}
{\bf dicembre 2011 -- dicembre 2019.}
Settore MAT/09 (Ricerca Operativa).


\subvoice{Professore associato presso la
          Facolt\`a di Ingegneria di Udine}
{\bf ottobre 2005 -- dicembre 2011.}
Settore MAT/09 (Ricerca Operativa).
Idoneit\`a ottenuta nel giugno 2003.


\subvoice{Ricercatore presso la
          Facolt\`a di Scienze di Povo (Trento)}
{\bf marzo 2001 -- ottobre 2005.}
Settore INF/01 (Informatica).


\subvoice{Ricercatore R1 presso l'I.R.S.T.}
{\bf Agosto 2000 -- febbraio 2001:}
inserito nel gruppo CBR (Case Based Reasoning, coordinato da Paolo Avesani)
della divisione SRA
(Sistemi di Ragionamento Automatico, diretta da Paolo Traverso) in IRST.
L'IRST (Istituto Ricerca Scientifica e Tecnologica)
\`e un organo dell'ITC (Istituto Trentino Cultura).\\

\subvoice{Posizioni temporanee presso Universit\`a ed Enti di Ricerca all'estero}
{\bf Agosto 99 -- ottobre 99:}
     Assistant Research Professor
     al  BRICS (Universit\`a di Aarhus, Denmark). 

\noindent
{\bf Aprile 2000 -- giugno 2000},
{\bf novembre 99 -- dicembre 99},
{\bf aprile 99 -- giugno 99},
{\bf novembre 98 -- dicembre 98:}
      Ho ricoperto, per un totale di 10 mesi,
      una posizione di ricerca su fondi DONET
      presso il centro di ricerca CWI di Amsterdam.
      Ero inserito nel gruppo PNA (Probability, Networks, Algorithms)
      sotto la guida dei professori Alexander Schrijver
      e Bert Gerards.\\
%mio primo periodo al CWI di Amsterdam, nel gruppo PNA (Probability, Networks, Algorithms) guidato dei professori Alexander Schrijver e Bert Gerards.
%Posizione di ricerca  su borsa Europea (fondi DONET).

\subvoice{Borsista post-dottorato}
{\bf Giugno 98 -- giugno 99:}
Per borsa
bandita dall'Universit\`a di Padova
ed usufruita
presso il Dipartimento di Matematica 
della stessa,
sotto la guida del prof.~Michele Conforti.\\



\newpage

\voice{{\LARGE Riviste Internazionali 2018--2022}}

\input{ListaRivisteInternazionali2018-2022}


%\newpage
\vspace{1.8mm}
\voice{{\LARGE Conferenze Internazionali 2018--2022 con Referee}}

\input{ListaConferenzeInternazionali2018-2022}






\vspace{1.2cm}
Verona \hspace{7.8cm} Romeo Rizzi


\end{document}

