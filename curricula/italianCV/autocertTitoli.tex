\documentclass[11pt]{article}
\usepackage[italian]{babel}

\textwidth 15.5cm
\textheight 23.8cm
\topmargin 0cm
\evensidemargin 0in
\oddsidemargin 0in

\pagestyle{empty}

\begin{document}

\begin{center}
   DICHIARAZIONI SOSTITUTIVE DI CERTIFICAZIONI \\
   (ai sensi dell'art. 46 del D.P.R. 445 del  28/12/2000) \\

\vspace{3mm}
   DICHIARAZIONI SOSTITUTIVE DELL'ATTO DI NOTORIET\'A \\
   (ai sensi dell'art. 47 del D.P.R. 445 del  28/12/2000)
\end{center}
 
\bigskip
                                  
\noindent
Io sottoscritto {\bf RIZZI ROMEO} codice fiscale n. {\bf RZZRMO67D20F187L},
nato a {\bf Mezzolombardo (TN)} il {\bf 20-04-67} sesso M
e residente in Martignano (TN) Via Bolleri n.~16/1,
consapevole 
delle sanzioni penali previste per il caso di dichiarazione 
mendace, cos\'\i\  come dall'art.~26 della Legge 04.01.1968 n.~15,
richiamato dall'art.~6, comma~2, del D.P.R. n.~403/1998

\begin{center}
 DICHIARA
\end{center}
sotto la propria responsabilità e consapevole delle sanzioni penali previste dall’art. 76 del D.P.R. 445 del  28/12/2000  in caso di dichiarazione mendace,

di possedere i seguenti titoli valutabili.\\


\section{Stato Famiglia}

\begin{center}
\begin{tabular}[c]{||l|l|l|l||}
 \hline \hline
    \multicolumn{1}{||c}{Cognome e nome} &
    \multicolumn{1}{|c}{Luogo di nascita} &
    \multicolumn{1}{|c}{Data di nascita} &
    \multicolumn{1}{|c||}{Rapporto di Parentela} \\
 \hline \hline
    RIZZI Romeo & Mezzolombardo & 20-4-67 & sottoscritto \\
 \hline
    MORONI Cristina & Trento & 22-1-69 & moglie \\
 \hline
    RIZZI Andrea & Trento & 3-12-94 & figlio \\
 \hline
    RIZZI Stefano & Trento & 3-10-2000 & figlio \\
 \hline \hline
\end{tabular}
\end{center}



\section{Titoli di Studio}

\begin{itemize}
\item[] (dicembre 1991) {\bf laurea in ingegneria elettronica},
      Politecnico di Milano. Media esami: 29/30. Voto finale: {\bf 100/100 e Lode};
\item[] (settembre 1997) titolo di {\bf dottore di ricerca},
         a seguito di un Dottorato in
         Matematica Computazionale ed Informatica Matematica (IX ciclo)
         presso il Dipartimento di Matematica Applicata
         dell'Universit\'a di Padova;
\end{itemize}


\section{Servizio Militare}
\begin{itemize}
   \item[] \hspace{-8.0mm}{\bf Assolto}: \ Incorporato il 16 novembre 92.
                          \ Congedato il 15 novembre 93.
\end{itemize}


\section{Abilitazioni}
\begin{itemize}
\vspace{-1.0mm}
   \item[]
      {\em Professione di Ingegnere} \ 
       - esame di stato: Milano, giugno 1992;
                    
\vspace{-1.0mm}
   \item[]
      {\em Insegnamento matematica per le superiori (047A)} \ 
       - concorso ordinario: Bolzano, marzo 2000;

\vspace{-1.0mm}
   \item[]
      {\em Insegnamento fisica per le superiori (048A)} \ 
       - concorso ordinario: Bolzano, maggio 2000.
\end{itemize}

\section{Diplomi}
\begin{itemize}
\vspace{-1.0mm}
   \item[]
      {\em diploma di operatore meccanografico} \ 
       - a seguito del periodo di leva: Bolzano, giugno 1993.
\end{itemize}


\section{Qualifica attuale}

\begin{itemize}
\item {\bf Professore associato presso la
          Facolt\`a di Ingegneria di Udine}
{\bf ottobre 2005 -- oggi.}
Settore MAT/09 (Ricerca Operativa).
Idoneit\`a ottenuta nel giugno 2003.


\item {\bf Ricercatore}, dal 01.03.2001 (e confermato dal 2004),
settore INF/01, presso la {\em Facolt\`a di Scienze MM.FF.NN.}
dell'{\em Universit\`a degli Studi di Trento}.
Afferente al {\em Dipartimento di Matematica} dal 01.03.2001 al
31.12.2001 ed al 
{\em Dipartimento di Informatica e Telecomunicazioni} dal 01.01.2002
ad oggi.

\end{itemize}


\section{Editor}

Area Editor per la rivista scientifica 4OR.


\section{Supplenze presso istituti di scuola superiore}

Ho svolto diversi periodi di supplenze temporanee
presso vari istituti superiori, sia prima che dopo la laurea
(prima della laurea solo corsi serali).
Tra questi, due incarichi annuali:
\begin{itemize}
   \item anno scolastico 97/98:
supplenza in ``Matematica'' ed in ``Matematica ed Informatica''
all' {{\sc I.P.C.} ``L.~Battisti''} 
di Trento.\\

   \vspace{-3.0mm}
   \item anno scolastico 89/90:
supplenza in ``Elettrotecnica'' ed in ``Misure Elettriche''
all' {{\sc I.T.I.S.} ``P.Hensenberger''}
di Monza.\\
\end{itemize} 

In totale,
ho svolto i seguenti periodi di supplenza:

\begin{center}
\begin{tabular}[c]{||c|p{0.80in}|p{1.25in}|p{1.25in}|p{1.25in}||}
 \hline \hline
  anno      & periodo & scuola & discipline di insegnamento & note\\
 \hline \hline
  89-90     & intero anno scolastico & I.T.I.S. Hensenberger (Monza)
            & (elettrotecnica) (misure elettriche)
            & prima della laurea (e solo serale) \\
 \hline \hline
  92-93     & dal 21/9/92 al 17/10/92
            & I.T.I. Marconi (Rovereto)
            & (informatica industriale) % codice: LIV 
              (matematica applicata) % codice: LXIV
            & nessuna \\
  92-93     & dal 26/10/92 al 14/11/92 & I.T.C. Martini (Mezzolombardo)
            & 038A (fisica) & nessuna \\
  92-93     & dal 15/11/92 al 10/06/93 & I.T.C. Martini (Mezzolombardo)
            & 038A (fisica)
            & nomina valida ai soli fini giuridici (servizio militare) \\
 \hline \hline
  93-94     & dal 13/10/93 al 18/11/93 & I.T.I.S. Buonarroti (Trento)
            & 035A (elettrotecnica e applicazioni) & nessuna \\
  93-94     & dal 12/2/94 al 26/2/94 & I.T.C. Martini (Mezzolombardo)
            & 048A (matematica applicata) & nessuna \\
 \hline \hline
  95-96     & dal 22/9/95 al 6/11/95 & I.T.I.S. Buonarroti (Trento)
            & 035A (elettrotecnica e applicazioni) & 1 giorno di assenza \\
 \hline \hline
  96-97   & dal 17/4/97 al 21/4/97 & I.P.C. Don Milani (Rovereto)
            & 042A (informatica) & nessuna \\
 \hline \hline
  97-98     & intero anno scolastico & I.P.C. Battisti (Trento)
            & 047A (matematica) (matematica ed informatica)
            & nessuna \\
 \hline \hline
  98-99   & dal 17/9/98 all' 1/10/98 & I.T.C.G. Floriani (Riva)
            & 048A (matematica applicata) & nessuna \\
  98-99   & dal 11/1/99 al 11/1/99 & I.T.C.G. Fontana (Rovereto)
            & 047A (matematica) & nessuna \\
 \hline \hline
  99-2000   & dal 15/1/00 al 31/3/00 & I.T.I.S. Buonarroti (Trento)
            & 047A (matematica) & nessuna \\
 \hline \hline
\end{tabular}
\end{center}


\section{Docenze Universitarie (anteriormente al mio inquadramento come ricercatore)}

\begin{itemize}
\item[] (Secondo semestre anno accademico 97/98)
     professore a contratto per 
     un corso integrativo di Programmazione Combinatoria
     nell'ambito del corso di Programmazione Matematica
     al Dipartimento di Matematica dell'Universit\`a di Trento;
\item[] (Secondo semestre anno accademico 96/97)
     esercitatore del corso di Analisi II
     per il Diploma di Ingegneria Informatica ed Automatica a Rovereto.
\end{itemize}


\section{Contratti di Ricerca}

\begin{itemize}
\item[] (Agosto 2000 -- Febbraio 2001)
     Contratto come ricercatore R1 presso l'I.R.S.T.
     (Istituto Ricerca Scientifica e Tecnologica).
     Ho lasciato quella posizione, non per scadenza del contratto,
     ma per assumere l'attuale ruolo
     di ricercatore universitario.
     L'IRST \'e un organo dell'ITC (Istituto Trentino Cultura);
\item[] (Agosto 99 -- Ottobre 99)
     Assistant Research Professor
     presso il  BRICS dell'Universit\'a di Aarhus (Denmark);
\item[] (Aprile 2000 -- Giugno 2000,
          Novembre 99 -- Dicembre 99,
          Aprile 99 -- Giugno 99,
          Novembre 98 -- Dicembre 98)
      ho ricoperto, per un totale di 10 mesi,
      delle posizioni temporanee su fondi DONET presso
      il gruppo PNA (Probability, Networks and Algorithms)
      del centro di ricerca CWI di Amsterdam;
\item[] (Giugno 98 -- Giugno 99)
      ho fruito per un anno di una borsa post-doc
      dell'Universit\'a di Padova
      presso il Dipartimento
      di Matematica dell'Univerit\'a di Padova.
\end{itemize}


\section{Contratti e Collaborazioni}

\begin{itemize}
\item[] (2004--08) arruolato dal Comitato Olimpico dell'AICA
come allenatore e selezionatore
della nazionale italiana per le edizioni 2004,05,06,07,08
delle {\em Olimpiadi di Informatica}.

\item[] (2001--08)
contratti con il {\em Liceo Scientifico Galileo Galilei}
per dei corsi di preparazione alle {\em Olimpiadi di Informatica}
e rivolti agli studenti interessati
di tutte le scuole superiori della provincia di Trento.
In seno all'attivit\`a di preparazione alle olimpiadi
si inseriscono anche,
partendo dal 2002, contratti
sia con l'{\em I.T.I.S.
Max Valier} di Bolzano che
con la {\em Sovrintendenza Scolastica Tedesca} di Bolzano
per docenze presso l'I.T.I.S.
Max Valier,
e dal 2006 dei contratti presso l'
{\em Istituto di Istruzione Bertrand Russel} di Cles.

\item[] (Aprile -- Luglio 2000)
     contratto con la {\em Libera Universit\`a di Bolzano}
     in merito al {\em Progetto Giornalino Virtuale}
     che ha visto allievi ed insegnanti di alcune scuole elementari
     in Alto-Adige
     inventare e redarre un loro simpatico giornalino in internet.
\item[] (Giugno 97 -- Aprile 98)
     contratto di prestazione professionale
     per la realizzazione di moduli software per il gruppo LEA
     presso il Dipartimento di Matematica dell'Universit\`a di Trento.
\end{itemize}


\section{Attivit\`a come ingegnere e Progetti}
\begin{itemize}
\item[] (Gennaio 94 -- Dicembre 96)
     iscritto all' Albo degli Ingegneri di Trento.
     In questo periodo ho presentato un progetto edile
     per un'abitazione in Vezzano (Trento).
     Tale progetto \'e stato approvato e realizzato.
\end{itemize}

\section{Partecipazione a progetti di ricerca}
   Ho preso parte ai seguenti progetti di ricerca,
   nazionali e locali.

\medskip

\begin{itemize}
\item 2001-03:
      Progetto CoFin RE-AL-WI-NE,
      Responsabile Nazionale: Alan  A.~Bertossi;
\item 2002-03:
      Progetto FIRB ``ADONIS'',
      Responsabile Uni-TN: R.~Battiti;
\item 2002-04:
      Progetto UNITN-PAT ``WILMA'',
      Responsabile Uni-TN: R.~Battiti;
\item 2005:
      Progetto Galileo (coord. parte italiana Romeo Rizzi,
                        coord. parte francese Guillaume Fertin).
\item 2007:Cofin. MIUR, (PRIN)
      (coord. naz. Prof. Vercellis, coord. loc. Lancia).
\end{itemize}


\section{Pubblicazioni Scientifiche su Rivista Nazionale}


\begin{enumerate}
  \item {\sc Romeo Rizzi},
   \newblock  Impaccando $T$-tagli e $T$-giunti,
   \newblock {\it Bollettino Sezione B dell'Unione Matematica Italiana},
   \newblock {Fascicolo speciale dedicato alle tesi di dottorato}
             (8) 1-A Suppl. (1998) 201-204.

\end{enumerate}

\section{Parti di Libro}

\begin{enumerate}
  \item {\sc Alan A.~Bertossi, Cristina M.~Pinotti, Romeo Rizzi, Phalguni Gupta},
   \newblock  Scalable algorithms for server allocation in infostations,
   \newblock  Chapter~20 in:
               Handbook of Research on Scalable Computing Technologies
   \newblock (K-C. Li,C-H. Hsu, L.T.~Yang, J.~Dongarra, H.~Zima Editors),
   \newblock  IGI Global,
   \newblock 2008.

  \item {\sc Alan A.~Bertossi, Cristina M.~Pinotti, Romeo Rizzi},
   \newblock  Scheduling data broadcasts on wireless channels:
               exact solutions and heuristics,
   \newblock  Chapter~73 in:
               Handbook of Approximation Algorithms and Metaheuristics
   \newblock (T.F. Gonzalez, Editor),
   \newblock  Taylor \& Francis Books (CRC Press),
   \newblock Chapman \& Hall/CRC, Boca Raton, 2007, pp.~73.1--73.16.

  \item {\sc Alan A.~Bertossi, Cristina M.~Pinotti, Romeo Rizzi, Anil M.~Shende},
   \newblock  Channel assignment in honeycomb networks,
   \newblock Theoretical computer science,  150--162,
   \newblock Lecture Notes in Comput. Sci., 2841,
   \newblock Springer, Berlin, 2003.

  \item {\sc Roberto Battiti, Alan A.~Bertossi, Romeo Rizzi},  
   \newblock  Randomized Greedy Algorithms 
              for the Hypergraph Partitioning Problem,
   \newblock Cap.~2, Vol.~43.
   \newblock Randomized Methods in Algorithm Design.
   \newblock DIMACS: Series in Discrete Mathematics
             and Theoretical Computer Science
   \newblock Pardalos P., Rajasekaran S., Rolim J. (a cura di),
             Providence, RI: American Mathematical Society.
   \newblock 1998. pp. 3-21.

\end{enumerate}


\section{Riviste Internazionali}

\begin{enumerate}
\vspace{-3.0mm}

 \vspace{1.2mm}
  \item {\sc Romeo Rizzi},
   \newblock   Approximating the Maximum $3$-Edge-Colorable Subgraph Problem,
   \newblock {\it Discrete Mathematics},
             accepted (2008).

  \item {\sc Guillaume Fertin, Romeo Rizzi, St\'ephane Vialette},
   \newblock  Finding Occurrences of Protein
              Complexes in Protein-Protein Interaction Graphs,
   \newblock {\it Journal of Discrete Algorithms},
             to appear (2008).

  \item {\sc Danny Hermelin, Dror Rawitz, Romeo Rizzi, St\'ephane Vialette},
   \newblock  The Minimum Substring Cover Problem,
   \newblock {\it Information and Computation},
   \newblock 206(11) (2008) 1303--1312.

  \item {\sc Stefano Benati, Romeo Rizzi},
   \newblock   The optimal statistical median of a convex set of arrays,
   \newblock {\it Journal of Global Optimization},
             to appear (2008),
             already published in online first mode (2008).

  \item {\sc Romeo Rizzi},
   \newblock   Minimum Weakly Fundamental Cycle Bases Are Hard To Find,
   \newblock {\it Algorithmica},
   \newblock 53(3) (2009) 402--424.

  \item {\sc Richard C.~Brewster, Pavol Hell, Romeo Rizzi},
   \newblock  Oriented star packings,
   \newblock {\it Journal of Combinatorial Theory, Series~B}
   \newblock  98 (2008) 558--576.

  \item {\sc Giuseppe Lancia, R.~Ravi, Romeo Rizzi},
   \newblock  Haplotyping for Disease Association: A Combinatorial Approach, 
   \newblock {\it IEEE Transactions on Computational Biology and Bioinformatics}
   \newblock  5(2) (2008) 245--251.

  \item {\sc Reuven Cohen, Liran Katzir, Romeo Rizzi},
   \newblock   On the Trade-off Between Energy and Multicast Efficiency in 802.16e-like Mobile Networks,
   \newblock {\it IEEE Transactions on Mobile Computing}
   \newblock  7(3) (2008) 346--357.
%   \newblock \\ - a previous version,
%                  with only Reuven Cohen and Romeo Rizzi as authors,
%                  was also accepted at Infocom~2006.

  \item {\sc Giuseppe Lancia, Franca Rinaldi, Romeo Rizzi},
   \newblock  Flipping letters to minimize the support of a string,
   \newblock  {\it International Journal of Foundations of Computer Science},
   \newblock  19(1) (2008) 5--17.

  \item {\sc Guillaume Blin, Cedric Chauve, Guillaume Fertin, Romeo Rizzi, St\'ephane Vialette},
   \newblock  Comparing Genomes with Duplications: A Computational Complexity Point of View.
   \newblock  IEEE/ACM Trans. Comput. Biology Bioinform.
   \newblock  4(4) (2007) 523--534.

  \item {\sc Michael Elkin, Christian Liebchen, Romeo Rizzi},
   \newblock  New length bounds for cycle bases,
   \newblock {\it Information Processing Letters}
   \newblock  104(5) (2007) 186--193.

  \item {\sc Francesco Maffioli, Romeo Rizzi, Stefano Benati},
   \newblock  Least and most colored bases,
   \newblock {\it Discrete Applied Mathematics}
   \newblock  155(15) (2007) 1958--1970.

  \item {\sc Stephen Finbow, Andrew King, Gary MacGillivray, Romeo Rizzi},
   \newblock  The firefighter problem for graphs of maximum degree three,
   \newblock {\it Discrete Mathematics}
   \newblock  307(16) (2007) 2094--2105.

  \item {\sc Christian Liebchen, Romeo Rizzi},
   \newblock  Classes of cycle bases,
   \newblock {\it Discrete Applied Mathematics}
   \newblock  155 (2007) 337--355.
  % gia inserito nel 2006

  \item {\sc Stefano Benati, Romeo Rizzi},
   \newblock  A mixed integer linear programming formulation
              of the optimal mean/Value-at-Risk portfolio problem,
   \newblock {\it European Journal of Operational Research}
   \newblock  176 (2007) 423--434.
  % gia inserito nel 2006

  \item {\sc Alessandro Mei, Romeo Rizzi},
   \newblock  Online Permutation Routing in
              Partitioned Optical Passive Star Networks,
   \newblock {\it IEEE Trans. Computers}
   \newblock  55(12) (2006) 1557--1571.

  \item {\sc Alessandro Mei, Romeo Rizzi},
   \newblock  Hypercube Computations on Partitioned Optical
              Passive Stars Networks,
   \newblock {\it IEEE Trans. Parallel Distrib. Syst.}
   \newblock  17(6) (2006) 497--507.
%   \newblock \\ - a preliminary version appeared in: HiPC 2003, 95--104.

  \item {\sc Romeo Rizzi},
   \newblock  Acyclically Pushable Bipartite Permutation Digraphs: an algorithm,
   \newblock {\it Discrete Mathematics}
   \newblock 306(12) (2006) 1177--1188.

  \item {\sc Romeo Rizzi, Marco Rospocher},
   \newblock  Covering partially directed graphs with directed paths,
   \newblock {\it Discrete Mathematics}
   \newblock 306(13) (2006) 1390--1404.

  \item {\sc Giuseppe Lancia, Romeo Rizzi},
   \newblock  A polynomial case of the parsimony haplotyping problem,
   \newblock {\it Oper. Res. Lett.}
   \newblock  34(3) (2006) 289--295.

  \item {\sc Guillaume Blin, Guillaume Fertin, Romeo Rizzi,
                  St\'ephane Vialette},
   \newblock What Makes the Arc-Preserving Subsequence Problem Hard?
   \newblock  {\it Transactions on Computational Systems Biology II}
   \newblock  LNCS vol.~3680 (2005) 1--36.
%   \newblock \\ - a draft version of this work appeared on:
%                  International Conference on Computational Science (2) 2005: 860-868.

  \item {\sc Vineet Bafna, Sorin Istrail, Giuseppe Lancia, Romeo Rizzi},
   \newblock  Polynomial and APX-hard cases of the Individual Haplotyping Problem,
   \newblock {\it Theoretical Computer Science}
   \newblock  335(1) (2005) 109--125.
%   \newblock \\ - a previous related work by the same authors
%               ``SNPs Problems: Complexity and Practical Algorithms''
%              was also accepted at WABI 2002.

  \item {\sc Zhi-Zhong Chen, Tao Jiang, Guohui Lin, Romeo Rizzi,
                  Jianjun Wen, Dong Xu, Ying Xu},
   \newblock  More Reliable Protein NMR Peak Assignment via Improved $2$-Interval Scheduling,
   \newblock {\it Journal of Computational Biology}
   \newblock 12(2) 2005 129--146.

  \item {\sc Christian Liebchen, Romeo Rizzi},
   \newblock  A greedy approach to compute a minimum cycle basis
              of a directed graph,
   \newblock {\it Information Processing Letters}
   \newblock  94(3) (2005) 107--112.
   %IPL3233
   %online via ScienceDirect:
   %http://authors.elsevier.com/TrackMyPaper.html?add_art=myarticles&trk_article=IPL3233&trk_mail=on&trk_surname=Liebchen

  \item {\sc Mauro Cettolo, Michele Vescovi, Romeo Rizzi},
   \newblock  Evaluation of BIC-based algorithms for audio segmentation,
   \newblock {\it Computer Speech \& Language}
   \newblock 19(2) (2005) 147--170.
%   \newblock Volume 19, Issue 2, April (2005) pages 147-170
%   \newblock \\ - a previous work by the same authors
%                ``A DP Algorithm for Speaker Change Detection'',
%               a starting point for this subsequent work,
%               had been accepted at Eurospeech 2003.

  \item {\sc Elia~Ardizzoni, Alan A.~Bertossi, Maria Cristina Pinotti,
                  Shashank Ramaprasad,  Romeo Rizzi,  Madhusudana V.S. Shashanka},
   \newblock  Optimal Skewed Data Allocation on Multiple Channels with Flat
              Broadcast per Channel,
   \newblock {\it IEEE Transactions on Computers}
   \newblock  54(5) (2005) 558--572.

  \item {\sc A.A. Bertossi, M.C. Pinotti, R. Rizzi, P. Gupta}, % Phalguni Gupta 
   \newblock  Allocating Servers in Infostations for Bounded Simultaneous Requests,
   \newblock {\it Journal of Parallel and Distributed Computing}
   \newblock  64 (2004) 1113--1126.
%   \newblock \\ - also accepted at IEEE Int'l Parallel and Distributed Processing Symposium, 2003.

  \item {\sc Alan A.~Bertossi, Cristina M.~Pinotti, Romeo Rizzi, Anil M.~Shende},
   \newblock  Channel Assignment for Interference Avoidance in Honeycomb Wireless Networks,
   \newblock {\it Journal of Parallel and Distributed Computing}
   \newblock  64 (2004) 1329--1344.

  \item {\sc Alberto Caprara, Andrea Lodi, Romeo Rizzi},
   \newblock  On $d$-Threshold Graphs and $d$-Dimensional Bin Packing,
   \newblock {\it Networks}
   \newblock  44(4) (2004) 266--280.
%   \newblock \\ - also accepted at 2003 Optimization Days.

  \item {\sc Alberto Caprara, Alessandro Panconesi, Romeo Rizzi},
   \newblock  Packing Cuts in Graphs,
   \newblock {\it Networks}
   \newblock  44(1) (2004) 1--11.
%   \newblock \\ - part of this work was published in the proceedings of ESA 2001.

  \item {\sc Giuseppe Lancia, Maria Cristina Pinotti, Romeo Rizzi},
   \newblock  Haplotyping Populations by Pure Parsimony: Complexity, 
              Exact, and Approximation Algorithms,
   \newblock {\it INFORMS J.~on Comp.}
   \newblock  16(4) (2004) 348--359.

  \item {\sc Michele Conforti, Romeo Rizzi},  
   \newblock  Combinatorial Optimization
              - Polyhedra and efficiency: A book review,
   \newblock {\it 4OR}
   \newblock 2(2) (2004) 153--159.

  \item {\sc Alberto Caprara, Alessandro Panconesi, Romeo Rizzi},
   \newblock  Packing Cycles in Undirected Graphs,
   \newblock {\it Journal of Algorithms}
   \newblock  48(1) (2003) 239--256.
%   \newblock \\ - part of this work was published in the proceedings of ESA 2001.
   % report DIT: NO - POLARIS: SI

  \item {\sc Romeo Rizzi},
   \newblock  On Rajagopalan and Vazirani's $\frac{3}{2}$-Approximation
              Bound for the Iterated $1$-Steiner Heuristic,
   \newblock {\it Information Processing Letters}
   \newblock  86(6) (2003) 335--338.
   %IPL2862
   %online via ScienceDirect:
   %http://www.sciencedirect.com/science?_ob=GatewayURL&_origin=AUTHORALERT&_method=citationSearch&_piikey=S0020019003002102&_version=1&md5=5603f0c51caeab0a09ec923ed91eccbe
   % report DIT: NO - POLARIS: SI

  \item {\sc Alessandro Mei, Romeo Rizzi},
   \newblock  Routing Permutations in Partitioned Optical Passive Stars Networks,
   \newblock {\it Journal of Parallel and Distributed Computing}
   \newblock  63(9) (2003) 847--852.
   \newblock \\ - also accepted at IPDPS 2002 where it received the {\bf Best Paper Award}.
   % report DIT: NO - POLARIS: SI

  \item {\sc Richard C.~Brewster, Romeo Rizzi},
   \newblock  On the complexity of digraph packings,
   \newblock {\it Information Processing Letters}
   \newblock  86(2) (2003) 101--106.
   %IPL2829
   %online via ScienceDirect:
   %http://www.sciencedirect.com/science?_ob=GatewayURL&_origin=AUGATEWAY&_method=citationSearch&_piikey=S0020019002004787&_version=1&md5=4550693a5fda4e6a7d92fb51803d3114
   % report DIT: NO - POLARIS: SI

  \item {\sc Romeo Rizzi},
   \newblock  A Simple Minimum $T$-Cut Algorithm,
   \newblock {\it Discrete Applied Mathematics}
   \newblock  129 (2003) 539--544.
   % report DIT: SI (senza dati rivista) - POLARIS: SI

  \item {\sc Richard C.~Brewster, Pavol Hell, Sarah H.~Pantel, Romeo Rizzi, Anders Yeo},
   \newblock  Packing paths in digraphs,
   \newblock {\it Journal of Graph Theory}
   \newblock  44(2) (2003) 81--94.
   %Published Online: 2 Sep 2003 DOI: 10.1002/jgt.10126
   % report DIT: NO - POLARIS: SI

  \item {\sc Romeo Rizzi},
   \newblock  Cycle cover property and $CPP=SCC$ property are not equivalent,
   \newblock {\it Discrete Mathematics}
   \newblock  259 (2002) 337--342.
   % report DIT: SI - POLARIS: SI

  \item {\sc Alberto Caprara, Romeo Rizzi},
   \newblock  Packing Triangles in Bounded Degree Graphs,
   \newblock {\it Information Processing Letters}
   \newblock  84(4) (2002) 175--180.
   % report DIT: SI - POLARIS: SI

  \item {\sc Romeo Rizzi},
   \newblock  Minimum $T$-cuts and optimal $T$-pairings,
   \newblock {\it Discrete Mathematics}
   \newblock  257(1) (2002) 177--181.
   % report DIT: SI - POLARIS: SI

  \item {\sc Romeo Rizzi},
   \newblock  Finding $1$-factors in bipartite regular graphs,
              and edge-coloring bipartite graphs,
   \newblock {\it SIAM Journal on Discrete Mathematics}
   \newblock  15(3) (2002) 283--288. 
   % report DIT: SI - POLARIS: SI

  \item {\sc Alberto Caprara, Romeo Rizzi},
   \newblock  Improved Approximation for Breakpoint Graph Decomposition
              and Sorting by Reversals,
   \newblock {\it Journal of Combinatorial Optimization}
   \newblock  6 (2002) 157--182.
   % report DIT: SI - POLARIS: SI

  \item {\sc Romeo Rizzi},
   \newblock  Complexity of Context-free Grammars with Exceptions,
              and the inadequacy of grammars as models for XML and SGML,
   \newblock {\it Markup Languages: Theory and Practice}
   \newblock  3(1) (2001) 107--116.
   % report DIT: SI - POLARIS: SI

  \item {\sc Alessandro Panconesi, Romeo Rizzi},
   \newblock  Some Simple Distributed Algorithms for Sparse Networks,
   \newblock {\it Distributed Computing}
   \newblock  14 (2001) 97--100.
   % report DIT: SI - POLARIS: SI

  \item {\sc Romeo Rizzi},
   \newblock  On the Recognition of $P_4$-Indifferent Graphs,
   \newblock {\it Discrete Mathematics}
   \newblock  239 (2001) 161--169.
   % report DIT: SI - POLARIS: SI

  \item {\sc Romeo Rizzi},
   \newblock  On $4$-connected graphs without even cycle decompositions,
   \newblock {\it Discrete Mathematics}
   \newblock  234 (2001) 181--186.
   % report DIT: SI - POLARIS: SI

  \item {\sc Romeo Rizzi},
   \newblock  Excluding a simple good pair approach to directed cuts,
   \newblock {\it Graphs and Combinatorics}
   \newblock  17 (2001) 741--744.
   % report DIT: SI - POLARIS: SI

  \item {\sc Michele Conforti, Romeo Rizzi},  
   \newblock  Shortest Paths in Conservative Graphs,
   \newblock {\it Discrete Mathematics}
   \newblock  226 (2001) 143--153.
%   \newblock \\ - part of this work
%                  was published in the proceedings of AIRO '96.
   % report DIT: SI - POLARIS: SI

  \item {\sc Romeo Rizzi},
   \newblock  A note on range-restricted circuit covers,
   \newblock {\it Graphs and Combinatorics}
   \newblock  16 (2000) 355--358.
   % report DIT: SI - POLARIS: SI

  \item {\sc Romeo Rizzi},
   \newblock  On minimizing symmetric set functions,
   \newblock {\it Combinatorica}
   \newblock  20(3) (2000) 445--450.
   % report DIT: SI - POLARIS: SI

  \item {\sc Romeo Rizzi},
   \newblock  A short proof of K\H{o}nig's matching theorem,
   \newblock {\it Journal of Graph Theory}
   \newblock  33(3) (2000) 138--139.
   % report DIT: SI - POLARIS: SI

  \item {\sc Ajai Kapoor, Romeo Rizzi},
   \newblock  Edge-coloring bipartite graphs,
   \newblock {\it Journal of Algorithms}
   \newblock  34(2) (2000) 390--396.
   % report DIT: SI - POLARIS: SI

  \item {\sc Romeo Rizzi},
   \newblock  Indecomposable $r$-graphs and some other counterexamples,
   \newblock {\it Journal of Graph Theory}
   \newblock  32(1) (1999) 1--15.
   % report DIT: SI - POLARIS: SI

  \item {\sc Alberto Caprara, Romeo Rizzi},  
   \newblock  Improving a Family of Approximation
              Algorithms to Edge Color Multigraphs,
   \newblock {\it Information Processing Letters}
   \newblock  68(1) (1998) 11--15.
   % report DIT: SI - POLARIS: SI

  \item {\sc Romeo Rizzi},
   \newblock  K\H{o}nig's Edge Coloring Theorem without augmenting paths,
   \newblock {\it Journal of Graph Theory}
   \newblock  29 (1998) 87.
   % report DIT: SI - POLARIS: SI

\end{enumerate}


\section{Conferenze}

\begin{enumerate}

  \item {G.~Fertin, D.~Hermelin, R.~Rizzi, S.~Vialette},
   \newblock Common Structured Patterns in Linear Graphs: Approximations and Combinatorics,
   \newblock 18th Symposium on Combinatorial Pattern Matching (CPM'07).
   \newblock LNCS~4580:241--252 (2007).

\vspace{-1.8mm}
  \item {G.~Brevier, R.~Rizzi, S.~Vialette},
   \newblock Pattern Matching in Protein-Protein Interaction Graphs,
   \newblock Proc.~16th International Symposium on Fundamentals
              of Computation Theory (FCT 2007).
   \newblock LNCS~4639:137--148 (2007).

\vspace{-1.8mm}
  \item {D.~Hermelin, D.~Rawitz, R.~Rizzi, S.~Vialette},
   \newblock The Minimum Substring Cover Problem,
   \newblock In, Christos Kaklamanis, Martin Skutella, editors,
   \newblock 5th Workshop on Approximation and Online Algorithms (WAOA'07).
   \newblock LNCS~4927:170--183 (2007).

\vspace{-1.8mm}
  \item {C.~Liebchen, G.~W\"unsch, E.~K\"ohler, A.~Reich, R.~Rizzi},
   \newblock Benchmarks for Strictly Fundamental Cycle Bases,
   \newblock WEA 2007.
   \newblock LNCS~4525:365-378 (2007).

\vspace{-1.8mm}
  \item {M.~Kubica, R.~Rizzi, S.~Vialette, T.~Wale\'n},
   \newblock Approximation of RNA Multiple Structural Alignment,
   \newblock 17th Symposium on Combinatorial Pattern Matching (CPM'06).
   \newblock LNCS~4009:211--222 (2006).

\vspace{-1.8mm}
  \item {G.~Lancia, F.~Rinaldi, R.~Rizzi},
   \newblock Flipping letters to minimize the support of a string,
   \newblock in The Prague Stringology Conference, PSC06,
   \newblock Stringology 9--17 (2006).

\vspace{-1.8mm}
  \item {\sc Reuven Cohen, Romeo Rizzi},
   \newblock   On the Trade-Off Between Energy and Multicast Efficiency in 802.16e-Like Mobile Networks,
   \newblock INFOCOM 2006.

\vspace{-1.8mm}
  \item {C.~Chauve, G.~Fertin, R.~Rizzi, S.~Vialette},
   \newblock Genomes containing duplicates are hard to compare,
   \newblock Int. Workshop on Bioinformatics Research and Applications (IWBRA).
   \newblock LNCS~3992:783--790 (2006).

\vspace{-1.8mm}
  \item {M.~Dalpasso, G.~Lancia and R.~Rizzi},
   \newblock The String Barcoding Problem is NP-Hard,
   \newblock in RECOMB Satellite on Comparative Genomics, (A.~Mc Lyshag and D.~Huson eds),
   \newblock Lecture Notes in Bioinformatics, Springer, 85--93, (2005). 

\vspace{-1.8mm}
  \item {G.~Fertin, R.~Rizzi, S.~Vialette},
   \newblock Finding Exact and Maximum Occurrences
of Protein Complexes in Protein-Protein Interaction Graphs,
   \newblock International Symposium on Mathematical Foundations of Computer Science (MFCS'05).
   \newblock LNCS~3618:328--339 (2005).

\vspace{-1.8mm}
  \item {A.~Mei, R.~Rizzi},
   \newblock Online Permutation Routing in Partitioned Optical Passive Star Networks,
   \newblock CoRR abs/cs/0502093. (2005).

\vspace{-1.8mm}
  \item {G.~Blin, G.~Fertin, R.~Rizzi, S.~Vialette},
   \newblock What Makes the Arc-Preserving Subsequence Problem Hard?
   \newblock 5th Int. Workshop on Bioinformatics Research and Applications (IWBRA'05).
   \newblock  LNCS~3515:860--868 (2005).

\vspace{-1.8mm}
  \item {G.~Blin, R.~Rizzi},
   \newblock Conserved Interval Distance Computation Between Non-trivial Genomes,
   \newblock COCOON 2005.
   \newblock  LNCS~3595:22--31 (2005).

\vspace{-1.8mm}
  \item {G.~Lancia, R.~Rizzi},
   \newblock Combinatorial Problems Arising in the Analysis of Human Polymorphisms,
   \newblock AIRO 2005, Camerino, 2005.

\vspace{-1.8mm}
  \item {G.~Lancia, F.~Rinaldi, R.~Rizzi},
   \newblock Reducing the k-mer diversity of a string,
   \newblock AIRO 2004, Lecce, 2004.

\vspace{-1.8mm}
  \item {G.~Blin, G.~Fertin, R.~Rizzi, S.~Vialette},
   \newblock Pattern Matching in Arc-Annotated Sequences: New Results for the APS Problem,
   \newblock 5th Journ\'ees Ouvertes de Biologie, Informatique et Math\'ematiques (JOBIM'04).
   \newblock Montr\'eal, Quebec. 2004.
   \newblock IEEE Computer Society.

\vspace{-1.8mm}
  \item {E.~Ardizzoni, A.A.~Bertossi, M.C.~Pinotti, R.~Rizzi},
   \newblock Comparing Algorithms for Data Broadcasting over Multiple Channels,
   \newblock Algorithms for Wirelss and Ad-hoc networks (ASWAN),
   \newblock Boston, USA, August 2004.

\vspace{-1.8mm}
  \item {A.A.~Bertossi, M.C.~Pinotti, S.~Ramaprasad, R.~Rizzi, M.V.S.~Shashanka},
   \newblock Optimal multi-channel data allocation with flat broadcast per channel,
   \newblock IEEE Int’l Parallel and Distributed Processing Symposium (IPDPS),
   \newblock Santa Fe, April 2004.

\vspace{-1.8mm}
  \item {Z-Z.~Chen, T.~Jiang, G.-H. Lin, R.~Rizzi, J.~Wen, D.~Xu, Y.~Xu},
   \newblock More Reliable Protein NMR Peak Assignment via Improved 2-Interval Scheduling,
   \newblock ESA 2003.
   \newblock LNCS~2832:580-592 (2003).

\vspace{-1.8mm}
  \item {M.~Cettolo, M.~Vescovi, R.~Rizzi},
   \newblock  A DP Algorithm  for Speaker Change Detection,
   \newblock Eurospeech 2003.

\vspace{-1.8mm}
  \item {A.~Mei, R.~Rizzi},
   \newblock Mapping Hypercube Computations onto Partitioned Optical Passive Star Networks,
   \newblock HiPC.
   \newblock  LNCS~2913:95--104 (2003).

\vspace{-1.8mm}
  \item {S.~Finbow, A.~King, G.~MacGillivray, R.~Rizzi},
   \newblock The Firefighter Problem for Graphs of Maximum Degree Three,
   \newblock EuroComb'03.

\vspace{-1.8mm}
  \item {A.A.~Bertossi, M.C.~Pinotti, R.~Rizzi, A.M.~Shende},
   \newblock Channel Assignment in Honeycomb Networks,
   \newblock 3rd ICTCS,
   \newblock Bertinoro, Italy, October 2003.

\vspace{-1.8mm}
  \item {A.A.~Bertossi, M.C.~Pinotti R.~Rizzi},
   \newblock Channel Assignment with Separation on Trees and Interval Graphs,
   \newblock 3rd Int’l Workshop on Wireless, Mobile and Ad Hoc Networks,
   \newblock (satellite workshop of IEEE IPDPS 2003),
   \newblock April 2003.

\vspace{-1.8mm}
  \item {A.A.~Bertossi, M.C.~Pinotti, R.~Rizzi, P.~Gupta},
   \newblock Allocating Servers in Infostations for Bounded Simultaneous Requests,
   \newblock IEEE Int’l Parallel and Distributed Processing Symposium (IPDPS),
   \newblock Nice, April 2003.

\vspace{-1.8mm}
  \item {A.A.~Bertossi, M.C.~Pinotti, R.~Rizzi},
   \newblock Channel assignment on strongly-simplicial graphs,
   \newblock IEEE WMAN,
   \newblock Nice, April 2003.

\vspace{-1.8mm}
  \item {R.~Rizzi, V.~Bafna, S.~Istrail, and G.~Lancia},
   \newblock Practical Algorithms and Fixed-parameter Tractability
   \newblock of the Single Individual SNP Haplotyping Problem,
   \newblock 2nd Workshop on Algorithms in Bioinformatics (WABI),
   \newblock LNCS 2452:29--43, 2002.

\vspace{-1.8mm}
  \item {A.~Mei, R.~Rizzi},
   \newblock Routing Permutations in Partitioned Optical Passive Star Networks,
   \newblock IPDPS 2002.

\vspace{-1.8mm}
  \item {A.~Caprara, A.~Panconesi, R.~Rizzi},
   \newblock  Packing Cycles and Cuts in Undirected Graphs,
   \newblock ESA 2001.
   \newblock LNCS 2161:512--523 (2001).

\vspace{-1.8mm}
  \item {R.~Rizzi},
   \newblock On minimizing symmetric set functions,
   \newblock Fourth Slovene International Conference
             in Graph Theory, 1999.

\vspace{-1.8mm}
  \item {M.~Conforti, R.~Rizzi},
   \newblock Shortest Paths in Conservative Graphs,
   \newblock AIRO '96, Perugia, 1996.

\end{enumerate}


\section{Tesi}

\begin{itemize}
\item[] {\bf tesi di laurea} (dicembre 1991) {\em laurea in ingegneria elettronica},
      Politecnico di Milano. Media esami: 29/30. Voto finale: {\bf 100/100 e Lode};
\item[] {\bf tesi di dottorato} (settembre 1997) {\em dottorato in
         Matematica Computazionale ed Informatica Matematica (IX ciclo)
         presso il Dipartimento di Matematica Applicata
         dell'Universit\'a di Padova.}
\end{itemize}



\section{Corsi tenuti presso la Facolt\`a di afferenza}

\begin{itemize}

\item (2000-01. Trento) Corso 
``Laboratorio di Algoritmi e Strutture Dati''
c.d.l. ``Informatica'', 
facolt\`a di Scienze MM.FF.NN., Universit\`a di Trento. 

\item (2001-02. Trento) Corso 
``Linear Programming'' 
nel corso di PhD ``International Graduate School of Information and Communication Technologies''
dell'Universit\`a di Trento. 

\item (2002-03. Trento) Corso 
``Computational Molecular Biology'' 
nel corso di PhD ``International Graduate School of Information and Communication Technologies''
dell'Universit\`a di Trento. 

\item (2002-03. Trento) Corso 
``Algoritmi e Strutture Dati I''
c.d.l. ``Informatica'', 
facolt\`a di Scienze MM.FF.NN., Universit\`a di Trento. 

\item (2002-03. Trento) Corso 
``Complessit\`a Computazionale''
c.d.l.s. ``Informatica'', 
facolt\`a di Scienze MM.FF.NN., Universit\`a di Trento. 

\item (2003-04. Trento) Corso 
``Computational Molecular Biology'' 
nel corso di PhD ``International Graduate School of Information and Communication Technologies''
dell'Universit\`a di Trento. 

\item (2003-04. Trento) Corso 
``Complessit\`a Computazionale''
c.d.l.s. ``Informatica'', 
facolt\`a di Scienze MM.FF.NN., Universit\`a di Trento. 

\item (2004-05. Trento) Corso 
``Complessit\`a Computazionale''
c.d.l.s. ``Informatica'', 
facolt\`a di Scienze MM.FF.NN., Universit\`a di Trento. 

\item (2005-06. Udine) Corso 
``Ricerca Operativa''
c.d.l. ``Ingegneria Gestionale Industriale'',
         ``Ingegneria Gestionale dell'Informazione'', 
facolt\`a di Ingegneria, Universit\`a di Udine. 

\item (2005-06. Udine) Corso 
``Matematica 2''
c.d.l. ``Architettura'',
facolt\`a di Architettura (presso Ingegneria), Universit\`a di Udine. 

\item (2006-07. Udine) Corso 
``Ricerca Operativa''
c.d.l. ``Ingegneria Gestionale Industriale'',
         ``Ingegneria Gestionale dell'Informazione'', 
facolt\`a di Ingegneria, Universit\`a di Udine. 

\item (2006-07. Udine) Corso 
``Matematica 2''
c.d.l. ``Architettura'',
facolt\`a di Architettura (presso Ingegneria), Universit\`a di Udine. 

\item (2006-07. Udine) Corso 
``Ricerca Operativa''
c.d.l.s. ``Architettura'',
facolt\`a di Architettura (presso Ingegneria), Universit\`a di Udine. 

\item (2007-08. Udine) Corso 
``Ricerca Operativa''
c.d.l. ``Ingegneria Gestionale Industriale'',
         ``Ingegneria Gestionale dell'Informazione'', 
facolt\`a di Ingegneria, Universit\`a di Udine. 

\item (2007-08. Udine) Corso 
``Matematica 2''
c.d.l. ``Architettura'',
facolt\`a di Architettura (presso Ingegneria), Universit\`a di Udine. 

\item (2007-08. Udine) Corso 
``Ricerca Operativa''
c.d.l.s. ``Architettura'',
facolt\`a di Architettura (presso Ingegneria), Universit\`a di Udine. 

\end{itemize}


\section{Corsi tenuti presso altri atenei}

\begin{itemize}

\item (2004-05. Perugia) Nella primavera 2005
ho tenuto un modulo in Programmazione Lineare
titolato ``Tecniche Anvanzate per la soluzione di problemi di ottimizzazione
combinatorica''
presso il Dipartimento di Matematica e Informatica
dell'Universit\`a di Perugia.
Titolarit\`a attribuita per chiara fama. 

\end{itemize}

\section{Advisor dottorato}

\begin{itemize}
\item (2001-corrente. Trento). Advisor dello studente di dottorato
  Marco Rospocher, International Graduate School in Information and
  Communication Technologies, Universit\`a di Trento. 
\end{itemize}


\section{Tesi di laurea}
\begin{itemize}

\item (2002. Trento) Relatore della tesi di laurea 
``{\em Algoritmi per la segmentazione
audio basati sul criterio di informazione bayesiano}''  
di Michele Vescovi, c.d.l. in Informatica, Facolt\`{a} 
di Scienze MM.FF.NN., Universit\`{a} di Trento.
Responsabile esterno: Mauro Cettolo - IRST. 

\item (2002, Trento) Relatore della tesi di laurea 
``{\em All-Pairs and Matching-like
Shortest Paths Algorithms}''
di Marco Rospocher, c.d.l. di Matematica, Facolt\`{a} 
di Scienze MM.FF.NN., Universit\`{a} di Trento. 

\item (2005, Trento) Relatore della tesi di laurea 
``{\em Gli Algoritmi di Ordinamento}''
di Paolo Zotti, c.d.l. in Informatica, Facolt\`{a} 
di Scienze MM.FF.NN., Universit\`{a} di Trento.

\item (2007, Trento) Correlatore della tesi di laurea 
``{\em Polynomial time instances for the IKHO problem}''
di Luca Nardin, c.d.l. in Informatica,
relatore: Prof.~Roberto Sebastiani, Facolt\`{a} 
di Scienze MM.FF.NN., Universit\`{a} di Trento.

\item (2007, Udine) Relatore della tesi di laurea 
``{\em Modellizzazione di un sistema di trasporti di una realt\`a aziendale}''
di Massimiliano Cossu, c.d.l. in Ingegneria Gestionale Industriale,
Facolt\`{a} di Ingegneria, Universit\`{a} di Udine.

\item (2007, Udine) Correlatore della tesi di laurea 
``{\em Algoritmi efficienti per il problema del minimum test collection}''
di Francesco Cafarelli, c.d.l. in Matematica,
relatore: Prof.~Franca Rinaldi,
Facolt\`{a} di Scienze MM.FF.NN., Universit\`{a} di Udine.

\item (2007, Trieste) Correlatore della tesi di laurea 
``{\em Approcci combinatorici al contenimento di fuochi}''
di Elia Calderan, c.d.l. in Matematica,
relatore: Prof.~Andrea Sgarro,
Facolt\`{a} di Scienze MM.FF.NN., Universit\`{a} di Trieste.

\end{itemize}


\section{Seminari tenuti}

Ho divulgato
i risultati dei miei lavori di ricerca mediante
seminari presso i seguenti istituti:
Laboratorio IMAG del CNRS in Grenoble (1995).
Laboratorio Leibniz dell'Universit\`a di Grenoble (1996, 2003).
Istituto IASI del C.N.R. in Roma (1997, 1999).
DEIS dell'Universit\`a di Bologna (1997, 1999, 2001),
DSI dell'Universit\`a di Bologna (2008).
Istituto di Ricerca CWI in Amsterdam (1998).
DEIS del Politecnico di Milano (2000, 2003),
Dipartimento di Informatica della Bicocca di Milano (2003, 2006, 2008).
Dipartimento di Informatica dell'Universit\`a di Verona (2008).
DSMI dell'Universit\`a di Reggio Emilia (2000).
Istituto IRST dell'ITC di Trento (2000, 2001).
Math. Dept. della Simon Freaser University di Vancouver (2000, 2003).
Math. Dept. della University of Victoria (2003).
Dipartimento di Elettronica del Politecnico di Torino (2003),
Engineering Dept. dell'Universit\`a di New Castle (2004).\\


\vspace{1.4cm}
               

Udine, 12 novembre 2008 \hspace{5.4cm} Firma

\end{document}
