\documentclass[10pt]{article}

\usepackage{latexsym}
\usepackage[italian]{babel}
\hyphenation{ra-gio-na-men-to}
\hyphenation{a-stra-zio-ne}
\hyphenation{pro-se-gui-men-to}


\textwidth 15.5cm
\textheight 23.8cm
\topmargin -1.4cm
\evensidemargin 0in
\oddsidemargin 0in

\newcommand{\voice}[1] { \bigskip \medskip \noindent {\Large \bf #1} \medskip\\ }
\newcommand{\subvoice}[1] { {\large \bf #1} \smallskip\\ }
\newcommand{\emp}[1] { {\em #1}\\ }

\begin{document}


\mbox{\ }
\vspace{-0.2cm} 
\begin{center}
{\Huge \sc {\bf Profilo in Biologia Computazionale \\ \vspace{4mm}}
               Rizzi Romeo}\\ \vspace{4mm}
               (primavera 2008)
\end{center}

\mbox{\ }
\vspace{-0.1cm}

{\bf Avvertenza:}
Questo documento intende raccogliere quei
soli elementi di attinenza al
settore della Biologia Computazionale
ed offre pertanto un CV mirato ma del tutto parziale.\\

 \vspace{2mm}



\voice{{\LARGE Progetti (per Biologia Computazionale)}}

Responsabile per la parte italiana
di un progetto Galileo (2004/2005).

Parte italiana: Trento, Udine, Milano Bicocca
(Romeo Rizzi, Giuseppe Lancia, Paola Bonizzoni,
Marco Rospocher, Riccardo Dondi).

Parte francese: Nantes, Parigi-Orsay
(Guillaume Fertin, Irena Rusu-Robini, St\'ephane Vialette,
J\'er\'emie Bourdon, Christine Sinoquet,
Guillaume Blin, Alban Mancheron, Christophe Moan).\\

Con una squadra allargata,
quest'anno abbiamo fatto domanda
sia per un Galileo che per un PICS.\\

 \vspace{2mm}



\voice{{\LARGE Riviste Internazionali (per Biologia Computazionale)}}

\begin{enumerate}
\vspace{-3.0mm}

 \vspace{1.2mm}
  \item {\sc Guillaume Fertin, Romeo Rizzi, St\'ephane Vialette},
   \newblock  Finding Occurrences of Protein
              Complexes in Protein-Protein Interaction Graphs.
   \newblock  accepted by {\it Journal of Discrete Algorithms} (2008).

  \item {\sc Danny Hermelin, Dror Rawitz, Romeo Rizzi, St\'ephane Vialette},
   \newblock  The Minimum Substring Cover Problem.
   \newblock  to appear {\it Information and Computation} (2008).

  \item {\sc Giuseppe Lancia, R.~Ravi, Romeo Rizzi},
   \newblock  Haplotyping for Disease Association: A Combinatorial Approach, 
   \newblock {\it IEEE Transactions on Computational Biology and Bioinformatics}
   \newblock  5(2): 245-251 (2008).

  \item {\sc Romeo Rizzi},
   \newblock   Minimum Weakly Fundamental Cycle Bases Are Hard To Find,
   \newblock {\it Algorithmica}, accepted,
             already published in online first mode.
%   \newblock  104(5) (2007) 186--193.

  \item {\sc Giuseppe Lancia, Franca Rinaldi, Romeo Rizzi},
   \newblock  Flipping letters to minimize the support of a string,
   \newblock  {\it International Journal of Foundations of Computer Science},
   \newblock  19(1) (2008) 5--17.

  \item {\sc Guillaume Blin, Cedric Chauve, Guillaume Fertin, Romeo Rizzi, St\'ephane Vialette},
   \newblock  Comparing Genomes with Duplications: A Computational Complexity Point of View.
   \newblock  IEEE/ACM Trans. Comput. Biology Bioinform.
   \newblock  4(4) (2007) 523--534.

  \item {\sc Christian Liebchen, Romeo Rizzi},
   \newblock  Classes of cycle bases,
   \newblock {\it Discrete Applied Mathematics}
   \newblock  155 (2007) 337--355.
  % gia inserito nel 2006

  \item {\sc Giuseppe Lancia, Romeo Rizzi},
   \newblock  A polynomial case of the parsimony haplotyping problem,
   \newblock {\it Oper. Res. Lett.}
   \newblock  34(3) (2006) 289--295.

  \item {\sc Guillaume Blin, Guillaume Fertin, Romeo Rizzi,
                  St\'ephane Vialette},
   \newblock What Makes the Arc-Preserving Subsequence Problem Hard?
   \newblock  {\it LNCS Transactions on Computational Systems Biology}
   \newblock  2 (2005) 1--36.

  \item {\sc Vineet Bafna, Sorin Istrail, Giuseppe Lancia, Romeo Rizzi},
   \newblock  Polynomial and APX--hard cases of the Individual Haplotyping Problem,
   \newblock {\it Theoretical Computer Science}
   \newblock  335(1) (2005) 109--125.

  \item {\sc Zhi-Zhong Chen, Tao Jiang, Guohui Lin, Romeo Rizzi,
                  Jianjun Wen, Dong Xu, Ying Xu},
% {\sc ZZ.~Chen, T.~Jiang, G.-H. Lin, R.~Rizzi, J.~Wen, D.~Xu, Y.~Xu},
   \newblock  More Reliable Protein NMR Peak Assignment via Improved
        $2$-Interval Scheduling.
   \newblock {\it Journal of Computational Biology}
   \newblock 12(2) 2005 129--146.

  \item {\sc Christian Liebchen, Romeo Rizzi},
   \newblock  A greedy approach to compute a minimum cycle basis
              of a directed graph,
   \newblock {\it Information Processing Letters}
   \newblock  94(3) (2005) 107--112.

  \item {\sc Giuseppe Lancia, Maria Cristina Pinotti, Romeo Rizzi},
   \newblock  Haplotyping Populations by Pure Parsimony: Complexity, 
              Exact, and Approximation Algorithms,
   \newblock {\it INFORMS J.~on Comp.}
   \newblock  16(4) (2004) 348--359.

  \item {\sc Alberto Caprara, Alessandro Panconesi, Romeo Rizzi},
   \newblock  Packing Cycles in Undirected Graphs,
   \newblock {\it Journal of Algorithms}
   \newblock  48(1) (2003) 239--256.


  \item {\sc Alberto Caprara, Romeo Rizzi},
   \newblock  Packing Triangles in Bounded Degree Graphs,
   \newblock {\it Information Processing Letters}
   \newblock  84(4) (2002) 175--180.

  \item {\sc Alberto Caprara, Romeo Rizzi},
   \newblock  Improved Approximation for Breakpoint Graph Decomposition
              and Sorting by Reversals,
   \newblock {\it Journal of Combinatorial Optimization}
   \newblock  6 (2002) 157--182.


\end{enumerate}


\voice{{\LARGE Conferenze (per Biologia Computazionale)}}

\begin{enumerate}
\vspace{-3.0mm}

 \vspace{1.2mm}

  \item {\sc Guillaume Fertin, Danny Hermelin, Romeo Rizzi, St\'ephane Vialette},
   \newblock  Common Structured Patterns in Linear Graphs,
   \newblock  Approximation and Combinatorics,
   \newblock CPM 2007,
   \newblock 241--252.

  \item {\sc Gaëlle Brevier, Romeo Rizzi, St\'ephane Vialette},
   \newblock  Pattern Matching in Protein-Protein Interaction Graphs. FCT 2007,
   \newblock 137--148.

  \item {\sc Danny Hermelin, Dror Rawitz, Romeo Rizzi, St\'ephane Vialette},
   \newblock  The Minimum Substring Cover Problem,
   \newblock WAOA 2007,
   \newblock  170--183.

  \item {\sc Christian Liebchen, Gregor W\"unsch, Ekkehard K\"ohler, Alexander Reich, Romeo Rizzi},
   \newblock  Benchmarks for Strictly Fundamental Cycle Bases,
   \newblock WEA 2007,
   \newblock  365--378.

  \item {\sc Marcin Kubica, Romeo Rizzi, St\'ephane Vialette, Tomasz Walen},
   \newblock  Approximation of RNA Multiple Structural Alignment,
   \newblock CPM 2006,
   \newblock  211--222.

  \item {\sc Cedric Chauve, Guillaume Fertin, Romeo Rizzi, St\'ephane Vialette},
   \newblock  Genomes Containing Duplicates Are Hard to Compare,
   \newblock International Conference on Computational Science (2) 2006,
   \newblock  783--790.

  \item {\sc Giuseppe Lancia, Franca Rinaldi, Romeo Rizzi},
   \newblock  Flipping letters to minimize the support of a string,
   \newblock Stringology 2006,
   \newblock  9--17.

  \item {\sc Guillaume Blin, Romeo Rizzi},
   \newblock  Conserved Interval Distance Computation Between Non-trivial Genomes. COCOON 2005,
   \newblock  22--31.

  \item {\sc Marcello Dalpasso, Giuseppe Lancia, Romeo Rizzi},
   \newblock  The String Barcoding Problem is NP-Hard,
   \newblock Comparative Genomics 2005,
   \newblock  88--96.
  \item {\sc Guillaume Blin, Guillaume Fertin, Romeo Rizzi, St\'ephane Vialette},
   \newblock  What Makes the Arc-Preserving Subsequence Problem Hard?,
   \newblock International Conference on Computational Science (2) 2005,
   \newblock  860--868.

  \item {\sc Guillaume Fertin, Romeo Rizzi, St\'ephane Vialette},
   \newblock  Finding Exact and Maximum Occurrences of Protein Complexes in Protein-Protein Interaction Graphs,
   \newblock MFCS 2005,
   \newblock  328--339.

\vspace{-1.8mm}
  \item {\sc Giuseppe Lancia, Romeo Rizzi},
   \newblock Combinatorial Problems Arising in the Analysis of Human Polymorphisms,
   \newblock AIRO 2005, Camerino, 2005.

\vspace{-1.8mm}
  \item {\sc Giuseppe Lancia, Franca Rinaldi, Romeo Rizzi},
   \newblock Reducing the k-mer diversity of a string,
   \newblock AIRO 2004, Lecce, 2004.

\vspace{-1.8mm}
  \item {\sc Guillaume Blin, Guillaume Fertin, Romeo Rizzi, St\'ephan Vialette},
   \newblock Pattern Matching in Arc-Annotated Sequences: New Results for the APS Problem.
   \newblock 5th Journ\'ees Ouvertes de Biologie, Informatique et Math\'ematiques (JOBIM'04).
   \newblock Montr\'eal, Quebec. 2004.
   \newblock IEEE Computer Society.

  \item {\sc Zhi-Zhong Chen, Tao Jiang, Guohui Lin, Romeo Rizzi, Jianjun Wen, Dong Xu, Ying Xu},
   \newblock  More Reliable Protein NMR Peak Assignment via Improved 2-Interval Scheduling,
   \newblock ESA 2003,
   \newblock  580--592.

  \item {\sc Romeo Rizzi, Vineet Bafna, Sorin Istrail, Giuseppe Lancia},
   \newblock  Practical Algorithms and Fixed-Parameter Tractability for the Single Individual SNP Haplotyping Problem,
   \newblock WABI 2002,
   \newblock  29--43.

  \item {\sc Alberto Caprara, Alessandro Panconesi, Romeo Rizzi},
   \newblock  Packing Cycles and Cuts in Undirected Graphs,
   \newblock ESA 2001,
   \newblock 512--523.
   \newblock \\ - almost in full also in: Twelfth Annual ACM-SIAM Symposium on Discrete Algorithms (Washington, DC, 2001).
   % report DIT: NO - POLARIS: SI



\end{enumerate}




\vspace{1.8mm}

\voice{{\LARGE Corsi in Biologia Computazionale}}

Nelle primavere 2002 e 2003
ho tenuto il corso
``Computational Molecular Biology: a Phd course''
per la PhD school del DIT dell'Universit\`a
di Trento.


\vspace{1.8mm}

\voice{{\LARGE Incarichi ed Onori (per Biologia Computazionale)}}

   - Ho tenuto il corso
    ``Algorithmic and Complexity issues in Structure Prediciton and/or Determination''
     alla Third International School on Biology,
     Computation and Information (BCI 2006).
     Dobbiaco (BZ), Italy, September 11-15, 2006.
    %home page della scuola: http://bioinf.dimi.uniud.it/bci2006/


   - Organizzatore, a fianco di Giuseppe Lancia,
     di una invited session in Computational Biology
     ad AIRO 2005. Camerino.

   - Invited speaker a BioInfoSummer 2004,
     tenutosi presso l'Australian National University in Canberra.

   - Invited speaker al
     ``Workshop on Cycle and Cut Bases'' (2008)
     tenutosi a T\"ubingen ed inserito nel quadro
     SPP 1126 (Algorithmik großer und komplexer Netzwerke).


\vspace{1.6cm}
Udine, primavera 2008. \hspace{6.8cm} Romeo Rizzi


\end{document}

