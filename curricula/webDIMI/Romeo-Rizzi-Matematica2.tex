\documentclass{article}
\usepackage{latexsym}
\usepackage[italian]{babel}
\usepackage{amsmath}
\usepackage{amsfonts}
\usepackage{amsthm}

\textwidth 15.5cm
\textheight 21.5cm
\topmargin 0cm
\evensidemargin 0in
\oddsidemargin 0in

\def\coNP{coN\!P}
\def\NP{N\!P}

\pagestyle{empty}
\begin{document}

\begin{center}
{\large
{\LARGE \bf \sc Matematica 2\\}
\vspace{2mm}
Corso di Laurea in Scienze dell'Architettura\\
\vspace{5mm}
docente, Prof.~Romeo Rizzi}
\end{center}

\section*{Presentazione del Corso}

   Il corso tratta l'analisi
   e l'uso di funzioni in pi\`u variabili
   ed \`e inteso ad estendere
   gli strumenti frutto dell'intuizione Cartesiana,
   come sviluppati ed assestati nel piano durante
   il propedeutico corso di Matematica~1,
   alle 3 dimensioni, e di fatto oltre.

   \`E un corso alla conquista dello spazio,
   e come tale di interesse per un Architetto
   che abbia piacere a concepire lo spazio
   anche da un punto di vista formale, ed ad avere tutti
   gli strumenti matematici e geometrici
   che possano consentire una pi\`u felice
   integrazione con le nuove tecnologie.

   Superato lo scoglio del passaggio da 2 a 3 dimensioni,
   lo studente realizzer\`a di poter avvalersi dello
   strumentario e delle nozioni sviluppate a Matematica~1,
   e come anche in parte lungo il percorso nelle scuole superiori,
   a spazi di un numero arbitrario di variabili.
   La visione offerta dall'analisi e dall'algebrizzazione della geometria
   si amplia con un potente guadagno in generalita`
   per le applicazioni di quanto gi\`a pazientemente sedimentato nel passato.
   Ed \`e comunque un corso di base che mira al fornire ed estendere
   competenze matematiche di base.

\section*{Programma del Corso}

\begin{itemize}

\item {\large $\mathbb{R}^n$ come spazio vettoriale}
 \begin{itemize}
    \item Cenni di insiemistica, prodotto Cartesiano,
          gli spazi $\mathbb{R}^n$;
    \item Vettori, somma, prodotto scalare, prodotto esterno;
    \item Coordinate polari, sferiche e cilindriche;
    \item Equazioni dei piani in $\mathbb{R}^3$;
    \item Equazioni delle rette in $\mathbb{R}^3$. 
 \end{itemize}

\item {\large Funzioni di pi\`u variabili}
 \begin{itemize}
    \item Grafico di una funzione in 2 variabili;
    \item Limiti e continuit\`a di funzioni in pi\`u variabili;
    \item Derivate parziali, funzioni di classe $\mathbf{C}^{1}$;
    \item Minimi/massimi locali;
    \item Teorema di Fermat per funzioni $\mathbf{C}^{1}$;
    \item Caratterizzazione dei punti di Estremo:
          condizioni necessarie, e condizioni sufficienti.
          Matrice Hessiana;

 \end{itemize}

\item {\large Integrazione in 2 e 3 variabili}
 \begin{itemize}
    \item Nozione di integrale di Riemann in 2 e 3 variabili (solo accennata);
    \item Integrazione su plurirettangoli e teorema di Fubini;
    \item Formula per il cambiamento di variabile
          negli integrali multipli. Lo Jacobiano.
 \end{itemize}

\item {\large Curve parametriche}
 \begin{itemize}
    \item Curve parametriche regolari in $\mathbb{R}^{n}$;
    \item Sostegno, curve equivalenti;
    \item Lunghezza di una curva e integrazione su curve;
    \item Ascissa curvilinea.
 \end{itemize}

\item {\large Superfici in $\mathbb{R}^{3}$}
 \begin{itemize}
    \item Superfici che sono grafico di una funzione reale di 2 variabili,
          con calcolo del piano tangente;
    \item Superfici che sono insiemi di livello di una funzione reale di 3 variabili,
          con calcolo del piano tangente;
    \item Superfici parametriche in $\mathbb{R}^{3}$ con formula per il calcolo
          dell'area e degli integrali di superfice. 
 \end{itemize}

\end{itemize}



\section*{Testo Adottato e di Riferimento}

\indent

   James Stewart,
   {\bf Calcolo, Funzioni di pi\'u variabili},
   2002 - APOGEO, ISBN: 88-7303-748-8.\\
         Questo testo, (e vi sar\'a sicuramente importante e forse
         necessario fare riferimento ad un qualche testo),
         \'e il completamento del testo di riferimento
         per il propedeutico corso di Matematica 1: Calcolo, Funzioni di una variabile.
         Ovviamente, lo studio dell'analisi nel caso di pi\'u variabili poggia pesantemente
         sullo studio del caso monovariabile.


\section*{Dispensa}

Una dispensa raccoglie {\bf tutti i temi d'esame} ed alcuni esercizi
per casa assegnati a partire dall'anno accademico 2003/2004.
La dispensa propone anche {\bf le correzioni} di alcuni temi
e pi\`u in generale vuole essere un documento
di accordo tra il docente e gli studenti su quello che deve
essere l'esame, sia nei contenuti che nelle modalit\`a.
Puoi sempre trovare l'ultima versione della dispensa
all'ufficio fotocopie presso i Rizzi od al sito:
          {\tt www.ten.dimi.uniud.it/\~\,rrizzi/classes/Mat2}
Copia della dispensa \`e inoltre
resa disponibile anche al sito:

         {\tt joshua.altervista.org/architettura/}

che, in modo pi\`u generico ed ampio, raccoglie materiale
per i vari corsi di Architettura.


\section*{Periodo}

Gennaio-marzo 2007.


\section*{Modalit\`a e svolgimento dell'Esame}

A fine corso uno scritto.
Ove pienamente positivo,
il voto dello scritto potr\`a essere direttamente registrato.
Di fatto, il 99\% di voi fa solamente lo scritto
(anche se una parte di voi lo prova pi\'u di una volta,
principalmente per raggiungere la sufficienza).
Nel preparare il tuo esame,
ti converr\`a prendere a riferimento la dispensa di cui sopra.
Essa riporta pi\`u nel dettaglio le regole per l'esame.

\section*{Prerequisiti}

    L'analisi nel caso monovariabile (e tutti i suoi prerequisiti).

\end{document}









